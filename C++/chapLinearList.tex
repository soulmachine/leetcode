\chapter{线性表}
这类题目考察线性表的操作,例如,数组,单链表,双向链表等。


\section{Add Two Numbers} %%%%%%%%%%%%%%%%%%%%%%%%%%%%%%
\label{sec:add-two-numbers}


\subsubsection{描述}
You are given two linked lists representing two non-negative numbers. The digits are stored in reverse order and each of their nodes contain a single digit. Add the two numbers and return it as a linked list.

Input: {\small \fontspec{Latin Modern Mono} (2 -> 4 -> 3) + (5 -> 6 -> 4)}

Output: {\small \fontspec{Latin Modern Mono} 7 -> 0 -> 8}


\subsubsection{分析}
跟Add Binary(见 \S \ref{sec:add-binary})很类似


\subsubsection{代码}
\begin{Code}
struct ListNode {
    int val;
    ListNode *next;
    ListNode(int x) : val(x), next(NULL) { }
};
 
//LeetCode, Add Two Numbers
//跟Add Binary 很类似
class Solution {
public:
    ListNode *addTwoNumbers(ListNode *l1, ListNode *l2) {
        ListNode* head = new ListNode(-1); // 头节点
        ListNode* pre = head;
        ListNode *pa = l1, *pb = l2;
        int carry = 0;
        while (pa != NULL || pb != NULL) {
            int av = pa == NULL ? 0 : pa->val;
            int bv = pb == NULL ? 0 : pb->val;
            ListNode* node = new ListNode((av + bv + carry) % 10);
            carry = (av + bv + carry) / 10;
            pre->next = node; // 尾插法
            pre = pre->next;
            pa = pa == NULL ? NULL : pa->next;
            pb = pb == NULL ? NULL : pb->next;
        }
        if (carry > 0)
            pre->next = new ListNode(1);
        pre = head->next;
        delete head;
        return pre;
    }
};
\end{Code}


\subsubsection{相关题目}

\begindot
\item Add Binary,见 \S \ref{sec:add-binary}
\myenddot

