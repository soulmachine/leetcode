\chapter{深度优先搜索}


\section{Palindrome Partitioning} %%%%%%%%%%%%%%%%%%%%%%%%%%%%%%
\label{sec:palindrome-partitioning}


\subsubsection{描述}
Given a string s, partition s such that every substring of the partition is a palindrome.

Return all possible palindrome partitioning of s.

For example, given \code{s = "aab"},
Return
\begin{Code}
  [
    ["aa","b"],
    ["a","a","b"]
  ]
\end{Code}


\subsubsection{分析}
在每一步都可以判断中间结果是否为合法结果,用回溯法。

一个长度为n的字符串,有n+1个地方可以砍断,每个地方可断可不断,因此复杂度为$O(2^{n+1})$


\subsubsection{代码}
\begin{Code}
//LeetCode, Palindrome Partitioning
class Solution {
public:
    vector<vector<string>> partition(string s) {
        vector<vector<string>> result;
        vector<string> output;  // 一个partition方案
        DFS(s, 0, output, result);
        return result;
    }
    // 搜索必须以s[start]开头的partition方案
    // 如果一个字符串长度为n,则可以插入n+1个隔板,复制度为O(2^{n+1})
    void DFS(string &s, int start, vector<string>& output,
            vector<vector<string>> &result) {
        if (start == s.size()) {
            result.push_back(output);
            return;
        }
        for (int i = start; i < s.size(); i++) {
            if (isPalindrome(s, start, i)) { // 从i位置砍一刀
                output.push_back(s.substr(start, i - start + 1));
                DFS(s, i + 1, output, result);  // 继续往下砍
                output.pop_back(); // 撤销上一个push_back的砍
            }
        }
    }
    bool isPalindrome(string &s, int start, int end) {
        while (start < end) {
            if (s[start] != s[end]) return false;
            start++;
            end--;
        }
        return true;
    }
};
\end{Code}


\subsubsection{相关题目}

\begindot
\item Palindrome Partitioning II,见 \S \ref{sec:palindrome-partitioning-ii}
\myenddot

