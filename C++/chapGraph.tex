\chapter{圖}

無向圖的節點定義如下:
\begin{Code}
// 無向圖的節點
struct UndirectedGraphNode {
    int label;
    vector<UndirectedGraphNode *> neighbors;
    UndirectedGraphNode(int x) : label(x) {};
};
\end{Code}


\section{Clone Graph} %%%%%%%%%%%%%%%%%%%%%%%%%%%%%%
\label{sec:clone-graph}


\subsubsection{描述}
Clone an undirected graph. Each node in the graph contains a \code{label} and a list of its \code{neighbours}.


OJ's undirected graph serialization:
Nodes are labeled uniquely.

We use \code{\#} as a separator for each node, and \code{,} as a separator for node label and each neighbour of the node.
As an example, consider the serialized graph \code{\{0,1,2\#1,2\#2,2\}}.

The graph has a total of three nodes, and therefore contains three parts as separated by \code{\#}.
\begin{enumerate}
\item First node is labeled as 0. Connect node 0 to both nodes 1 and 2.
\item Second node is labeled as 1. Connect node 1 to node 2.
\item Third node is labeled as 2. Connect node 2 to node 2 (itself), thus forming a self-cycle.
\end{enumerate}

Visually, the graph looks like the following:
\begin{Code}
       1
      / \
     /   \
    0 --- 2
         / \
         \_/
\end{Code}


\subsubsection{分析}
廣度優先遍歷或深度優先遍歷都可以。


\subsubsection{DFS}
\begin{Code}
// LeetCode, Clone Graph
// DFS,時間複雜度O(n),空間複雜度O(n)
class Solution {
public:
    UndirectedGraphNode *cloneGraph(const UndirectedGraphNode *node) {
        if(node == nullptr) return nullptr;
        // key is original node,value is copied node
        unordered_map<const UndirectedGraphNode *,
                            UndirectedGraphNode *> copied;
        clone(node, copied);
        return copied[node];
    }
private:
    // DFS
    static UndirectedGraphNode* clone(const UndirectedGraphNode *node,
            unordered_map<const UndirectedGraphNode *,
            UndirectedGraphNode *> &copied) {
        // a copy already exists
        if (copied.find(node) != copied.end()) return copied[node];

        UndirectedGraphNode *new_node = new UndirectedGraphNode(node->label);
        copied[node] = new_node;
        for (auto nbr : node->neighbors)
            new_node->neighbors.push_back(clone(nbr, copied));
        return new_node;
    }
};
\end{Code}


\subsubsection{BFS}
\begin{Code}
// LeetCode, Clone Graph
// BFS,時間複雜度O(n),空間複雜度O(n)
class Solution {
public:
    UndirectedGraphNode *cloneGraph(const UndirectedGraphNode *node) {
        if (node == nullptr) return nullptr;
        // key is original node,value is copied node
        unordered_map<const UndirectedGraphNode *,
                            UndirectedGraphNode *> copied;
        // each node in queue is already copied itself
        // but neighbors are not copied yet
        queue<const UndirectedGraphNode *> q;
        q.push(node);
        copied[node] = new UndirectedGraphNode(node->label);
        while (!q.empty()) {
            const UndirectedGraphNode *cur = q.front();
            q.pop();
            for (auto nbr : cur->neighbors) {
                // a copy already exists
                if (copied.find(nbr) != copied.end()) {
                    copied[cur]->neighbors.push_back(copied[nbr]);
                } else {
                    UndirectedGraphNode *new_node =
                            new UndirectedGraphNode(nbr->label);
                    copied[nbr] = new_node;
                    copied[cur]->neighbors.push_back(new_node);
                    q.push(nbr);
                }
            }
        }
        return copied[node];
    }
};
\end{Code}


\subsubsection{相關題目}
\begindot
\item 無
\myenddot

\section{Network Delay Time} %%%%%%%%%%%%%%%%%%%%%%%%%%%%%%
\label{sec:network-delay-time}


\subsubsection{描述}
There are N network nodes, labelled 1 to N.

Given times, a list of travel times as directed edges times[i] = (u, v, w), where u is the source node, v is the target node, and w is the time it takes for a signal to travel from source to target.

Now, we send a signal from a certain node K. How long will it take for all nodes to receive the signal? If it is impossible, return -1.

\begin{center}
\includegraphics[width=100pt]{network-delay-time.png}\\
\figcaption{Rotate Image}\label{fig:network-delay-time}
\end{center}

\begin{Code}
Input: times = [[2,1,1],[2,3,1],[3,4,1]], N = 4, K = 2
Output: 2
\end{Code}

\begin{enumerate}
\item N will be in the range [l, 100]
\item K will be in the range [l, N]
\item The length of times will be in the range [l, 6000].
\item All edges times[i] = (u, v, w) will have l <= u, v <= N and 0 <= w <= 100.
\end{enumerate}
\subsubsection{分析}
廣度優先遍歷
Use Djikstra Algorithm. Find a minium path from one source to one end.


\subsubsection{Djikstra}
\begin{Code}
// LeetCode, Clone Graph
// DFS with graph,時間複雜度O(ELogV),空間複雜度O(V^2)
class Solution {
public:
    struct Node {
        int m_node;
        int m_dist;

        Node (int node, int dist)
            : m_node(node), m_dist(dist) {}
    };
    int networkDelayTime(vector<vector<int>>& times, int N, int K) {
        // create graph
        unordered_map<int, list<Node>> graph;  // key: nodeNumber value: nodes
        for (const auto& t : times) // startNode: t[0], endNode: t[1], dist: t[2]
            graph[t[0]].push_back(Node(t[1], t[2]));

        unordered_map<int, int> dist; // key: nodeNumber value: dist
        multimap<int, int> dq; // djikstra queue, key: dist value: nodeNumber

        // do until no more node in queue
        dq.insert(make_pair(0, K));
        while (!dq.empty()) {
            pair<int, int> cur = *dq.begin();
            dq.erase(dq.begin());

            // filter out if shortest path is found
            if (dist.find(cur.second) != dist.end()) continue;
            // record the shortest parth
            dist.insert(make_pair(cur.second, cur.first));
            // push the neighbour to dq
            auto it = graph.find(cur.second);
            if (it == graph.end()) continue;
            for (const auto& nei : it->second) {
                if (dist.find(nei.m_node) != dist.end()) continue;
                dq.insert(make_pair(nei.m_dist + cur.first, nei.m_node));
            }
        }

        // pick the max dist, return -1 if not all node is visited
        if ((int)dist.size() == N)
            return max_element(dist.begin(), dist.end()
                                , [&](const pair<int, int>& first, const pair<int, int>& second)
                                {
                                    return first.second < second.second;
                                })->second;
        else
            return -1;
    }
};

\end{Code}
\subsubsection{Djikstra}
\begin{Code}
// LeetCode, Clone Graph
// DFS with adjacency list,時間複雜度O(ELogV),空間複雜度O(V^2)
class Solution {
public:
    int networkDelayTime(vector<vector<int>>& times, int N, int K)
    {
        // Build adjacency list
        vector<vector<pair<int, int>>> adjList(N + 1); // index 0 is dummy for easy implementation
        for (const auto& v : times)
            adjList[v[0]].emplace_back(v[1], v[2]);

        vector<int> dist(N + 1, INT_MAX);
        // initialize pq and source vertex distance
        dist[K] = 0;
        set<pair<int, int>> pq {{dist[K], K}};
        while (!pq.empty()) {
            const auto u = pq.begin()->second;
            pq.erase(pq.begin()); // pop the top, i.e., first element in set
            for (const auto& [v, w]: adjList[u]) {
                if (dist[u] + w < dist[v]) { // if edge can be relaxed
                    pq.erase({dist[v], v}); // remove old info
                    dist[v] = dist[u] + w;  // update distance
                    pq.emplace(dist[v], v); // insert new info
                }
            }
        }

        const int longestTime = *max_element(dist.begin() + 1, dist.end());
        return longestTime == INT_MAX ? -1 : longestTime;
    }
};
\end{Code}

\subsubsection{相關題目}
\begindot
\item 無
\myenddot
