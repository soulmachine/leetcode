\chapter{Google}
Google interview questions
\newline

\section{Odd Even Jump} %%%%%%%%%%%%%%%%%%%%%%%%%%%%%%
\label{sec:odd-even-jump}


\subsubsection{描述}
You are given an integer array A.  From some starting index, you can make a series of jumps.  The (1st, 3rd, 5th, ...) jumps in the series are called odd numbered jumps, and the (2nd, 4th, 6th, ...) jumps in the series are called even numbered jumps.

You may from index i jump forward to index j (with i < j) in the following way:

During odd numbered jumps (ie. jumps 1, 3, 5, ...), you jump to the index j such that A[i] <= A[j] and A[j] is the smallest possible value.  If there are multiple such indexes j, you can only jump to the smallest such index j.
During even numbered jumps (ie. jumps 2, 4, 6, ...), you jump to the index j such that A[i] >= A[j] and A[j] is the largest possible value.  If there are multiple such indexes j, you can only jump to the smallest such index j.
(It may be the case that for some index i, there are no legal jumps.)
A starting index is good if, starting from that index, you can reach the end of the array (index A.length - 1) by jumping some number of times (possibly 0 or more than once.)

Return the number of good starting indexes.

Example 1:
\begin{Code}
Input: [10,13,12,14,15]
Output: 2
Explanation:
From starting index i = 0, we can jump to i = 2 (since A[2] is the smallest among A[1], A[2], A[3], A[4] that is greater or equal to A[0]), then we can't jump any more.
From starting index i = 1 and i = 2, we can jump to i = 3, then we can't jump any more.
From starting index i = 3, we can jump to i = 4, so we've reached the end.
From starting index i = 4, we've reached the end already.
In total, there are 2 different starting indexes (i = 3, i = 4) where we can reach the end with some number of jumps.
\end{Code}

Example 2:
\begin{Code}
Input: [2,3,1,1,4]
Output: 3
Explanation:
From starting index i = 0, we make jumps to i = 1, i = 2, i = 3:

During our 1st jump (odd numbered), we first jump to i = 1 because A[1] is the smallest value in (A[1], A[2], A[3], A[4]) that is greater than or equal to A[0].

During our 2nd jump (even numbered), we jump from i = 1 to i = 2 because A[2] is the largest value in (A[2], A[3], A[4]) that is less than or equal to A[1].  A[3] is also the largest value, but 2 is a smaller index, so we can only jump to i = 2 and not i = 3.

During our 3rd jump (odd numbered), we jump from i = 2 to i = 3 because A[3] is the smallest value in (A[3], A[4]) that is greater than or equal to A[2].

We can't jump from i = 3 to i = 4, so the starting index i = 0 is not good.

In a similar manner, we can deduce that:
From starting index i = 1, we jump to i = 4, so we reach the end.
From starting index i = 2, we jump to i = 3, and then we can't jump anymore.
From starting index i = 3, we jump to i = 4, so we reach the end.
From starting index i = 4, we are already at the end.
In total, there are 3 different starting indexes (i = 1, i = 3, i = 4) where we can reach the end with some number of jumps.
\end{Code}

Example 3:
\begin{Code}
Input: [5,1,3,4,2]
Output: 3
Explanation:
We can reach the end from starting indexes 1, 2, and 4.
\end{Code}


\subsubsection{動規 - Monotonic Stack}
\begin{Code}
// 思路: 先求得單和雙數往後跳的目的地,再用動規,得知由某點能否跳往目的地
// 時間複雜度O(nlogn),空間複雜度O(n)
class Solution {
public:
    int oddEvenJumps(vector<int>& A) {
        vector<pair<int, int>> cache; cache.reserve(A.size());
        for (int i = 0; i < A.size(); i++)
            cache.emplace_back(A[i], i);
        // 找尋增大順序的下標
        sort(cache.begin(), cache.end()
             , [&](const auto& left, const auto& right)
             {
                 if (left.first == right.first)
                     return left.second < right.second;
                 else
                     return left.first < right.first;
             });
        vector<int> sortedIndex;
        sortedIndex.reserve(cache.size());
        for_each(cache.begin(), cache.end()
                , [&](const auto& ele) { sortedIndex.push_back(ele.second); } );
        // 求得單數跳躍的下標
        vector<int> oddNext = GetNext(sortedIndex);

        // 找尋減少順序的下標
        sort(cache.begin(), cache.end()
             , [&](const auto& left, const auto& right)
             {
                 if (left.first == right.first)
                     return left.second < right.second;
                 else
                     return left.first > right.first;
             });
        sortedIndex.clear();
        sortedIndex.reserve(cache.size());
        for_each(cache.begin(), cache.end()
                , [&](const auto& ele) { sortedIndex.push_back(ele.second); } );
        // 求得雙數跳躍的下標
        vector<int> evenNext = GetNext(sortedIndex);

        // 設 odd[i]: 單數跳躍時可以由 i 到 N
        vector<bool> odd(A.size(), false);
        // 設 even[i]: 雙數跳躍時可以由 i 到 N
        vector<bool> even(A.size(), false);
        odd.back() = even.back() = true;

        // odd 的總數便是答案
        for (int i = A.size() - 1; i >= 0; i--)
        {
            if (oddNext[i] != -1)
                odd[i] = even[oddNext[i]];
            if (evenNext[i] != -1)
                even[i] = odd[evenNext[i]];
        }

        int result = 0;
        for (const auto& o : odd)
            if (o) result++;

        return result;
    }
private:
    vector<int> GetNext(const vector<int>& inputIndex)
    {
        // 使用了 Monotonic Stack
        stack<int> cache;
        vector<int> result(inputIndex.size(), -1); // -1 代表沒法再往後跳躍

        for (const auto& index : inputIndex)
        {
            while (!cache.empty() && cache.top() < index)
            {
                result[cache.top()] = index;
                cache.pop();
            }
            cache.push(index);
        }

        return result;
    }
};
\end{Code}

\subsubsection{動規 - Map}
\begin{Code}
// 思路: 利用 BST 由尾至頭歷遍,找出每一個對應的跳躍目的地
// 時間複雜度O(nlogn),空間複雜度O(n)
class Solution {
public:
    int oddEvenJumps(vector<int>& A) {
        int N = A.size();
        // 設 odd[i]: 單數跳躍時可以由 i 到 N
        vector<bool> odd(A.size(), false);
        // 設 even[i]: 雙數跳躍時可以由 i 到 N
        vector<bool> even(A.size(), false);
        odd.back() = even.back() = true;

        map<int, int> cache; // key: A[i] value: i
        cache.emplace(A[N-1], N-1);

        for (int i = N - 2; i >= 0; i--)
        {
            auto it = cache.find(A[i]);
            if (it != cache.end())
            {
                odd[i] = even[it->second];
                even[i] = odd[it->second];
            }
            else
            {
                auto bigger = GetBigger(cache, A[i]);
                auto smaller = GetSmaller(cache, A[i]);

                if (bigger != cache.end())
                    odd[i] = even[bigger->second];
                if (smaller != cache.end())
                    even[i] = odd[smaller->second];
            }
            cache[A[i]] = i;
        }

        // odd 的總數便是答案
        int result = 0;
        for (const auto& o : odd)
            if (o) result++;

        return result;
    }
private:
    template <class MyMap, class T>
        typename MyMap::iterator GetBigger(MyMap& cache, T val)
    {
        auto bigger = cache.lower_bound(val);
        if (bigger == cache.end())
            return bigger;
        else
            return bigger;
    }
    template <class MyMap, class T>
        typename MyMap::iterator GetSmaller(MyMap& cache, T val)
    {
        auto smaller = cache.lower_bound(val);
        if (smaller == cache.begin())
            return cache.end();
        else
            return prev(smaller);
    }
};
\end{Code}

\subsubsection{動規 - Multimap}
\begin{Code}
// 思路: 利用 BST 由尾至頭歷遍,找出每一個對應的跳躍目的地
// 時間複雜度O(nlogn),空間複雜度O(n)
class Solution {
public:
    int oddEvenJumps(vector<int>& A) {
        int N = A.size();
        // 設 odd[i]: 單數跳躍時可以由 i 到 N
        vector<bool> odd(A.size(), false);
        // 設 even[i]: 雙數跳躍時可以由 i 到 N
        vector<bool> even(A.size(), false);
        odd.back() = even.back() = true;

        multimap<int, int> cache; // key: A[i] value: i
        cache.emplace(A[N-1], N-1);

        for (int i = N - 2; i >= 0; i--)
        {
            auto it = cache.find(A[i]);
            if (it != cache.end())
            {
                // 找尋同值最細下標
                it = GetSmallestIndex(cache, it);
                odd[i] = even[it->second];
                even[i] = odd[it->second];
            }
            else
            {
                auto bigger = GetBigger(cache, A[i]);
                auto smaller = GetSmaller(cache, A[i]);

                if (smaller != cache.end())
                    even[i] = odd[smaller->second];
                if (bigger != cache.end())
                    odd[i] = even[bigger->second];
            }
            cache.emplace(A[i], i);
        }

        // odd 的總數便是答案
        int result = 0;
        for (const auto& o : odd)
            if (o) result++;

        return result;
    }
private:
    template <class MyMap, class T>
        typename MyMap::iterator GetBigger(MyMap& cache, T val)
    {
        auto bigger = cache.lower_bound(val);
        if (bigger == cache.end())
            return bigger;
        else
            return GetSmallestIndex(cache, bigger); // 找尋同值最細下標
    }
    template <class MyMap, class T>
        typename MyMap::iterator GetSmaller(MyMap& cache, T val)
    {
        auto smaller = cache.lower_bound(val);
        if (smaller == cache.begin())
            return cache.end();
        else
            return GetSmallestIndex(cache, prev(smaller)); // 找尋同值最細下標
    }
    template <class MyMap, class MapIT>
        MapIT GetSmallestIndex(MyMap& cache, MapIT target)
    {
        auto range = cache.equal_range(target->first);
            // 找尋同值最細下標
            int minIndex = INT_MAX;
            for (auto j = range.first; j != range.second; j++)
            {
                if (minIndex > j->second)
                {
                    target = j;
                    minIndex = j->second;
                }
            }

        return target;
    }
};
\end{Code}

\section{Course Schedule II} %%%%%%%%%%%%%%%%%%%%%%%%%%%%%%
\label{sec:course-schedule-ii}


\subsubsection{描述}
There are a total of n courses you have to take, labeled from 0 to n-1.

Some courses may have prerequisites, for example to take course 0 you have to first take course 1, which is expressed as a pair: [0,1]

Given the total number of courses and a list of prerequisite pairs, return the ordering of courses you should take to finish all courses.

There may be multiple correct orders, you just need to return one of them. If it is impossible to finish all courses, return an empty array.

Example 1:
\begin{Code}
Input: 2, [[1,0]]
Output: [0,1]
Explanation: There are a total of 2 courses to take.
             To take course 1 you should have finished
             course 0. So the correct course order is [0,1] .
\end{Code}

Example 2:
\begin{Code}
Input: 4, [[1,0],[2,0],[3,1],[3,2]]
Output: [0,1,2,3] or [0,2,1,3]
Explanation: There are a total of 4 courses to take.
             To take course 3 you should have finished both courses 1 and 2.
             Both courses 1 and 2 should be taken after you finished course 0.
             So one correct course order is [0,1,2,3].
             Another correct ordering is [0,2,1,3] .
\end{Code}

Note:
\begindot
\item The input prerequisites is a graph represented by a list of edges, not adjacency matrices. Read more about how a graph is represented.
\item You may assume that there are no duplicate edges in the input prerequisites.
\myenddot

\subsubsection{DFS - Topological Sort}
\begin{Code}
// 時間複雜度O(n),空間複雜度O(n)
class Solution {
public:
    vector<int> findOrder(int numCourses, vector<vector<int>>& prerequisites) {
        vector<int> result; result.reserve(numCourses);

        // Topological Sort
        // 先造出 dependency graph
        // 也記低有什麼 course 出現過
        unordered_map<int, list<int>> dGraph;
        unordered_set<int> seenCourse;
        for (const auto& p : prerequisites)
        {
            dGraph[p[0]].push_back(p[1]);
            seenCourse.insert(p[0]);
            seenCourse.insert(p[1]);
        }

        // 先補上在 graph 中沒有出現的 course
        for (int i = 0; i < numCourses; i++)
            if (seenCourse.count(i) == 0) result.push_back(i);

        // 利用 DFS 造出答案
        unordered_map<int, bool> visited;

        for (const auto& [k, v] : dGraph)
        {
            if (!DFS(dGraph, k, result, visited)) return vector<int>();
        }

        return result;
    }
private:
    bool DFS(const unordered_map<int, list<int>>& dGraph, int k
             , vector<int>& result, unordered_map<int, bool>& visited)
    {
        if (visited.find(k) != visited.end()) return visited[k];

        visited[k] = false;

        auto it = dGraph.find(k);
        if (it != dGraph.end())
        {
            for (const auto& nei : it->second)
            {
                if (!DFS(dGraph, nei, result, visited)) return false;
            }
        }

        visited[k] = true;
        result.push_back(k);

        return true;
    }
};
\end{Code}

\section{Longest Increasing Path in a Matrix} %%%%%%%%%%%%%%%%%%%%%%%%%%%%%%
\label{sec:longest-increasing-path-in-a-matrix}


\subsubsection{描述}
Given an integer matrix, find the length of the longest increasing path.

From each cell, you can either move to four directions: left, right, up or down. You may NOT move diagonally or move outside of the boundary (i.e. wrap-around is not allowed).

Example 1:
\begin{Code}
Input: nums =
[
  [9,9,4],
  [6,6,8],
  [2,1,1]
]
Output: 4
Explanation: The longest increasing path is [1, 2, 6, 9].
\end{Code}

Example 2:
\begin{Code}
Input: nums =
[
  [3,4,5],
  [3,2,6],
  [2,2,1]
]
Output: 4
Explanation: The longest increasing path is [3, 4, 5, 6]. Moving diagonally is not allowed.
\end{Code}


\subsubsection{備忘錄法}
\begin{Code}
// 時間複雜度O(m*n),空間複雜度O(m*n)
class Solution {
public:
    int longestIncreasingPath(vector<vector<int>>& matrix) {
        m_M = matrix.size();
        if (m_M == 0) return 0;
        m_N = matrix[0].size();
        if (m_N == 0) return 0;

        vector<vector<bool>> visited(m_M, vector<bool>(m_N, false));
        // 設 cache[i][j] 為 i,j 的最長答案
        // 這個答案是由 0 開始總加的,所以 DFS 也要由 0 開始總加
        vector<vector<int>> cache(m_M, vector<int>(m_N, -1));
        int maxLen = INT_MIN;
        for (int i = 0; i < m_M; i++)
        {
            for (int j = 0; j < m_N; j++)
            {
                maxLen = max(maxLen, DFS(i, j, matrix, cache, visited));
            }
        }


        return maxLen;
    }
private:
    const vector<pair<int, int>> directions{{1,0}, {0,1}, {-1,0}, {0,-1}};
    int DFS(int i, int j, const vector<vector<int>>& matrix
            , vector<vector<int>>& cache, vector<vector<bool>>& visited)
    {
        if (cache[i][j] != -1) return cache[i][j];

        visited[i][j] = true;
        int maxLen = 0;
        for (int d = 0; d < 4; d++)
        {
            int newI = i + directions[d].first;
            int newJ = j + directions[d].second;
            if (newI >= 0 && newI < m_M && newJ >= 0 && newJ < m_N
                && !visited[newI][newJ]
                && matrix[i][j] > matrix[newI][newJ])// 由大至細地找尋答案 (由細至大也可以)
                maxLen = max(maxLen, DFS(newI, newJ, matrix, cache, visited));
        }
        visited[i][j] = false;

        return cache[i][j] = maxLen + 1; // 注意: 由細至大保存中途答案
    }
private:
    int m_M;
    int m_N;
};
\end{Code}

\section{Count Complete Tree Nodes} %%%%%%%%%%%%%%%%%%%%%%%%%%%%%%
\label{sec:count-complete-tree-nodes}


\subsubsection{描述}
Given a complete binary tree, count the number of nodes.

Note:

Definition of a complete binary tree from Wikipedia:
In a complete binary tree every level, except possibly the last, is completely filled, and all nodes in the last level are as far left as possible. It can have between 1 and 2h nodes inclusive at the last level h.

Example:
\begin{Code}
Input:
    1
   / \
  2   3
 / \  /
4  5 6

Output: 6
\end{Code}


\subsubsection{暴力}
\begin{Code}
// 時間複雜度O(n),空間複雜度O(n)
class Solution {
public:
    int countNodes(TreeNode* root) {
        if (root == nullptr) return 0;

        return 1 + countNodes(root->left) + countNodes(root->right);
    }
};
\end{Code}

\subsubsection{Binary Search}
\begin{Code}
// 時間複雜度O(d^2),空間複雜度O(n)
class Solution {
public:
    int countNodes(TreeNode* root) {
        // 若果樹沒有元素
        if (root == nullptr) return 0;

        int d = ComputeDepth(root);
        // 若果樹只有一個元素
        if (d == 0) return 1;

        // left: 為最低層的最左的 node index
        // right: 為最低層的最右的 node index
        int left = 1; int right = pow(2, d) - 1;
        while (left <= right)
        {
            int pivot = (left + right) / 2;
            if (IsExist(pivot, d, root))
                left = pivot + 1;
            else
                right = pivot - 1;
        }

        return pow(2, d) - 1 + left;
    }
private:
    int ComputeDepth(TreeNode *root)
    {
        int d = 0;
        while (root->left)
        {
            root = root->left;
            d++;
        }
        return d;
    }
    bool IsExist(int idx, int d, TreeNode* node)
    {
        int left = 0; int right = pow(2, d) - 1;
        for (int i = 0; i < d; i++)
        {
            int pivot = (left + right) / 2;
            if (idx <= pivot)
            {
                node = node->left;
                right = pivot;
            }
            else
            {
                node = node->right;
                left = ++pivot;
            }
        }
        return node != nullptr;
    }
};
\end{Code}

\section{Flip Equivalent Binary Trees} %%%%%%%%%%%%%%%%%%%%%%%%%%%%%%
\label{sec:flip-equivalent-binary-trees}


\subsubsection{描述}
For a binary tree T, we can define a flip operation as follows: choose any node, and swap the left and right child subtrees.

A binary tree X is flip equivalent to a binary tree Y if and only if we can make X equal to Y after some number of flip operations.

Write a function that determines whether two binary trees are flip equivalent.  The trees are given by root nodes root1 and root2.

Example 1:
\begin{Code}
Input:
       1                 1
      /  \              / \
     3    2            2   3
      \  / \          / \  /
      6 4   5        4  5 6
           / \         / \
          8   7       7   8

Input: root1 = [1,2,3,4,5,6,null,null,null,7,8], root2 = [1,3,2,null,6,4,5,null,null,null,null,8,7]
Output: true
Explanation: We flipped at nodes with values 1, 3, and 5.
\end{Code}

Note:
\begindot
\item Each tree will have at most 100 nodes.
\item Each value in each tree will be a unique integer in the range [0, 99]
\myenddot


\subsubsection{遞歸}
\begin{Code}
// 時間複雜度O(min(N1, N2)),空間複雜度O(min(H1, H2))
class Solution {
public:
    bool flipEquiv(TreeNode* root1, TreeNode* root2) {
        if (root1 == nullptr) return root2 == nullptr;
        if (root2 == nullptr) return root1 == nullptr;

        return root1->val == root2->val
            && (flipEquiv(root1->right, root2->left) && flipEquiv(root1->left, root2->right)
                || flipEquiv(root1->right, root2->right) && flipEquiv(root1->left, root2->left));
    }
};
\end{Code}

\subsubsection{Canonical Traversal}
\begin{Code}
// 時間複雜度O(min(N1, N2)),空間複雜度O(min(H1, H2))
class Solution {
public:
    bool flipEquiv(TreeNode* root1, TreeNode* root2) {
        list<TreeNode*> vals1;
        list<TreeNode*> vals2;

        // 由細至大把二叉樹放到鏈表中
        DFS(root1, vals1);
        DFS(root2, vals2);

        // 當兩個表是一樣,回 true
        return vals1.size() == vals2.size()
            && equal(vals1.begin(), vals1.end(), vals2.begin()
                     , [](const auto& left, const auto& right)
                     {
                         if (left == nullptr && right == nullptr) return true;
                         else if (left == nullptr || right == nullptr) return false;
                         else return left->val == right->val;
                     });
    }
private:
    void DFS(TreeNode *root, list<TreeNode*>& vals)
    {
        if (root == nullptr) return;

        vals.push_back(root);

        int L = root->left == nullptr ? -1 : root->left->val;
        int R = root->right == nullptr ? -1 : root->right->val;

        if (L < R)
        {
            DFS(root->left, vals);
            DFS(root->right, vals);
        }
        else
        {
            DFS(root->right, vals);
            DFS(root->left, vals);
        }

        vals.push_back(nullptr);
    }
};
\end{Code}

\section{Diameter of Binary Tree} %%%%%%%%%%%%%%%%%%%%%%%%%%%%%%
\label{sec:diameter-of-binary-tree}


\subsubsection{描述}
Given a binary tree, you need to compute the length of the diameter of the tree. The diameter of a binary tree is the length of the longest path between any two nodes in a tree. This path may or may not pass through the root.

Example:
Given a binary tree
\begin{Code}
          1
         / \
        2   3
       / \     
      4   5    

Return 3, which is the length of the path [4,2,1,3] or [5,2,1,3].

Note: The length of path between two nodes is represented by the number of edges between them.
\end{Code}


\subsubsection{遞歸}
\begin{Code}
// 時間複雜度O(n),空間複雜度O(1)
class Solution {
public:
    int diameterOfBinaryTree(TreeNode* root) {
        m_result = 1;
        DFS(root);
        return m_result - 1;
    }
private:
    int DFS(TreeNode *root)
    {
        if (root == nullptr) return 0;

        int L = DFS(root->left);
        int R = DFS(root->right);

        m_result = max(m_result, L+R+1); // 由左至右總加起來
        return max(L, R) + 1; // 注意: 只取最長的子樹
    }
private:
    int m_result;
};
\end{Code}

\section{Evaluate Division} %%%%%%%%%%%%%%%%%%%%%%%%%%%%%%
\label{sec:evaluate-division}


\subsubsection{描述}
Equations are given in the format A / B = k, where A and B are variables represented as strings, and k is a real number (floating point number). Given some queries, return the answers. If the answer does not exist, return -1.0.

Example:
Given a / b = 2.0, b / c = 3.0.
queries are: a / c = ?, b / a = ?, a / e = ?, a / a = ?, x / x = ? .
return [6.0, 0.5, -1.0, 1.0, -1.0 ].

\begin{Code}
  The input is: vector<pair<string, string>> equations, vector<double>& values
  , vector<pair<string, string>> queries , where equations.size() == values.size()
  , and the values are positive. This represents the equations. Return vector<double>.
\end{Code}

According to the example above:

\begin{Code}
equations = [ ["a", "b"], ["b", "c"] ],
values = [2.0, 3.0],
queries = [ ["a", "c"], ["b", "a"], ["a", "e"], ["a", "a"], ["x", "x"] ]. 
\end{Code}

The input is always valid. You may assume that evaluating the queries will result in no division by zero and there is no contradiction.

\subsubsection{遞歸 - DFS}
\begin{Code}
// 時間複雜度O(n),空間複雜度O(n)
class Solution {
public:
    vector<double> calcEquation(vector<vector<string>>& equations
                                , vector<double>& values
                                , vector<vector<string>>& queries){
        // 造圖
        unordered_map<string,vector<pair<double,string>>> g;
        for (int i = 0; i < equations.size(); i++)
        {
            string& strA = equations[i][0];
            string& strB = equations[i][1];
            // 記錄 A / B
            g[strA].emplace_back(values[i], strB);
            // 記錄 B / A
            g[strB].emplace_back(1 / values[i], strA);
        }

        vector<double> result;
        for (int i = 0; i < queries.size(); i++)
        {
            const string& qStart = queries[i][0];
            const string& qEnd = queries[i][1];
            // 若整個除數關係沒有記錄
            if (g.find(qStart) == g.end() || g.find(qEnd) == g.end())
                result.push_back(-1.0);
            else
            {
                // 若有記錄
                set<string> visited;
                int ansLen = result.size();
                DFS(qStart, qEnd, 1, visited, result, g);
                if (result.size() == ansLen)
                    result.push_back(-1); // 若找不到答案
            }
        }
        return result ;
    }
private:
    void DFS(string start, string end, double cost
             , set<string>& visited
             , vector<double>& result
             , unordered_map<string,vector<pair<double,string>>>& g)
    {
        visited.insert(start);

        if (start == end)
        {
            result.push_back(cost);
            return;
        }

        for (const auto& [num, neigbor] : g[start])
        {
            if (visited.find(neigbor) != visited.end()) continue;

            int ansLen = result.size();
            DFS(neigbor, end, cost * num, visited, result, g);
            if (ansLen != result.size()) return; // 已找到答案,剪枝
        }
    }
};
\end{Code}

\section{Cracking the Safe} %%%%%%%%%%%%%%%%%%%%%%%%%%%%%%
\label{sec:cracking-the-safe}


\subsubsection{描述}
There is a box protected by a password. The password is a sequence of n digits where each digit can be one of the first k digits 0, 1, ..., k-1.

While entering a password, the last n digits entered will automatically be matched against the correct password.

For example, assuming the correct password is "345", if you type "012345", the box will open because the correct password matches the suffix of the entered password.

Return any password of minimum length that is guaranteed to open the box at some point of entering it.

Example 1:
\begin{Code}
Input: n = 1, k = 2
Output: "01"
Note: "10" will be accepted too.
\end{Code}

Example 2:
\begin{Code}
Input: n = 2, k = 2
Output: "00110"
Note: "01100", "10011", "11001" will be accepted too.
\end{Code}

Note:
\begindot
\item n will be in the range (1, 4).
\item k will be in the range (1, 10).
\item pow(k, n) will be at most 4096.
\myenddot

\subsubsection{遞歸 - DFS - Hierholzer's Algorithm}
\begin{Code}
// 時間複雜度O(n * k^n),空間複雜度O(n * k^n)
// 利用 Hierholzer's Algorithm 歷遍所有 edges 並記錄所有的 nodes
// Hierholzer's Algorithm 用來找尋 Eulerian Path
class Solution {
public:
    string crackSafe(int n, int k) {
        if (n == 1 && k == 1) return "0";
        // seen 用來存放見過的 node
        unordered_set<string> seen;
        string result;

        // 準備開始點
        string start; start.append(n-1, '0');

        DFS(start, k, seen, result);
        result += start;

        return result;
    }
private:
    void DFS(const string& node, int k, unordered_set<string>& seen, string& result)
    {
        // 這是 post order travel 的做法
        for (int x = 0; x < k; x++)
        {
            // nei 是下一個歷遍的 node + edge
            // x 為 edge, 0, 1, 2 ... k-1
            string nei = node + to_string(x);
            if (seen.find(nei) == seen.end())
            {
                seen.insert(nei);
                // nei.substr(1) 可以提取下一個 node
                // 例子: 01(上一個 node) -> 010(edge) -> 10(下一個 node)
                DFS(nei.substr(1), k, seen, result);
                result += to_string(x); // 完成後,記低 edge
            }
        }
    }
};
\end{Code}

