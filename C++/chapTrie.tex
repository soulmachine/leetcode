\chapter{字首樹 Trie}
Things related to Trie data structure.
\newline

\section{Implement Trie (Prefix Tree)} %%%%%%%%%%%%%%%%%%%%%%%%%%%%%%
\label{sec:implement-trie}


\subsubsection{描述}
Implement a trie with insert, search, and startsWith methods.

Example:
\begin{Code}
Trie trie = new Trie();

trie.insert("apple");
trie.search("apple");   // returns true
trie.search("app");     // returns false
trie.startsWith("app"); // returns true
trie.insert("app");   
trie.search("app");     // returns true
\end{Code}

Note:
\begindot
\item You may assume that all inputs are consist of lowercase letters a-z.
\item All inputs are guaranteed to be non-empty strings.
\myenddot


\subsubsection{代碼}
\begin{Code}
// 時間複雜度O(),空間複雜度O()
struct TrieNode {
    bool flag;
    map<char, TrieNode*> next;
};
class Trie {
private:
    TrieNode* root;

public:
    /** Initialize your data structure here. */
    Trie() {
        root = new TrieNode();
    }

    /** Inserts a word into the trie. */
    void insert(string word) {
        TrieNode* p = root;
        for (int i = 0; i < word.length(); ++i) {
            if ((p->next).count(word[i]) <= 0) {
                // insert a new node if the path does not exist
                (p->next).insert(make_pair(word[i], new TrieNode()));
            }
            p = (p->next)[word[i]];
        }
        p->flag = true;
    }

    /** Returns if the word is in the trie. */
    bool search(string word) {
        TrieNode* p = root;
        for (int i = 0; i < word.length(); ++i) {
            if ((p->next).count(word[i]) <= 0) {
                return false;
            }
            p = (p->next)[word[i]];
        }
        return p->flag;
    }

    /** Returns if there is any word in the trie that starts with the given prefix. */
    bool startsWith(string prefix) {
        TrieNode* p = root;
        for (int i = 0; i < prefix.length(); ++i) {
            if ((p->next).count(prefix[i]) <= 0) {
                return false;
            }
            p = (p->next)[prefix[i]];
        }
        return true;
    }
};

/**
 * Your Trie object will be instantiated and called as such:
 * Trie obj = new Trie();
 * obj.insert(word);
 * bool param_2 = obj.search(word);
 * bool param_3 = obj.startsWith(prefix);
 */
\end{Code}

\section{Map Sum Pairs} %%%%%%%%%%%%%%%%%%%%%%%%%%%%%%
\label{sec:map-sum-pairs}


\subsubsection{描述}
Implement a MapSum class with insert, and sum methods.

For the method insert, you'll be given a pair of (string, integer). The string represents the key and the integer represents the value. If the key already existed, then the original key-value pair will be overridden to the new one.

For the method sum, you'll be given a string representing the prefix, and you need to return the sum of all the pairs' value whose key starts with the prefix.

Example 1:
\begin{Code}
Input: insert("apple", 3), Output: Null
Input: sum("ap"), Output: 3
Input: insert("app", 2), Output: Null
Input: sum("ap"), Output: 5
\end{Code}

\subsubsection{分析}
1. 暴力,先記低每一個加入的文字和其數值,當要求總加時,歷遍所有

2. 利用字首樹


\subsubsection{字首樹}
\begin{Code}
// 時間複雜度O(k),空間複雜度O(n)
class MapSum {
public:
    /** Initialize your data structure here. */
    MapSum() {

    }

    void insert(string key, int val) {
        TrieNode *p = &m_root;
        // delta 是關鍵,用來解決 value replacement 的問題。
        int delta = val - m_cache[key];
        m_cache[key] = val;
        for (const auto& c : key)
        {
            if ((p->m_next).count(c) <= 0)
                (p->m_next).emplace(c, new TrieNode());

            p = (p->m_next)[c];
            p->m_val += delta;
        }
    }

    int sum(string prefix) {
        TrieNode *p = &m_root;

        for (const char& c : prefix)
        {
            if ((p->m_next).count(c) <= 0)
                return 0;

            p = (p->m_next)[c];
        }

        return p->m_val;
    }
private:
    TrieNode m_root;
    unordered_map<string, int> m_cache;
};

/**
 * Your MapSum object will be instantiated and called as such:
 * MapSum* obj = new MapSum();
 * obj->insert(key,val);
 * int param_2 = obj->sum(prefix);
 */
\end{Code}

\section{Replace Words} %%%%%%%%%%%%%%%%%%%%%%%%%%%%%%
\label{sec:replace-words}


\subsubsection{描述}
In English, we have a concept called root, which can be followed by some other words to form another longer word - let's call this word successor. For example, the root an, followed by other, which can form another word another.

Now, given a dictionary consisting of many roots and a sentence. You need to replace all the successor in the sentence with the root forming it. If a successor has many roots can form it, replace it with the root with the shortest length.

You need to output the sentence after the replacement.

Example 1:
\begin{Code}
Input: dict = ["cat","bat","rat"], sentence = "the cattle was rattled by the battery"
Output: "the cat was rat by the bat"
\end{Code}

Constraints:
\begindot
\item The input will only have lower-case letters.
\item 1 <= dict.length <= 1000
\item 1 <= dict[i].length <= 100
\item 1 <= sentence words number <= 1000
\item 1 <= sentence words length <= 1000
\myenddot

\subsubsection{暴力計算}
\begin{Code}
// 時間複雜度O(sum of all w^2),空間複雜度O(n)
// 先記低全部 prefix,然後每個文子,每個 substring 都檢查一次並且取代原本的文字。
class Solution {
public:
    string replaceWords(vector<string>& dict, string sentence) {
        unordered_set<string> prefixDict(dict.begin(), dict.end());

        string result;
        bool isFirst = true;
        for (const string& word : GetVec(sentence))
        {
            string tmp;
            for (const char& w : word)
            {
                tmp += w;
                if (prefixDict.count(tmp) > 0)
                    break;
            }
            if (isFirst)
                result += tmp;
            else
                result += " " + tmp;
            isFirst = false;
        }

        return result;
    }
private:
    vector<string> GetVec(const string& sentence)
    {
        vector<string> result;
        for (auto i = sentence.begin(); i != sentence.end(); )
        {
            i = find_if(i, sentence.end(), [&](const char& c){ return (c >= 'a' && c <= 'z'); });
            if (i == sentence.end()) return result;

            auto j = find_if(i, sentence.end(), [&](const char& c){ return c == ' '; });
            result.push_back(string(i, j));
            i = j;
        }
        return result;
    }
};
\end{Code}

\subsubsection{Tier}
\begin{Code}
// 時間複雜度O(k),空間複雜度O(n)
struct TrieNode
{
    bool m_isWord;
    unordered_map<char, TrieNode*> m_next;

    TrieNode () { m_isWord = false; };
};
class Solution {
public:
    string replaceWords(vector<string>& dict, string sentence) {
        // 造字首樹
        TrieNode root;
        for (const string& s : dict)
        {
            TrieNode *p = &root;
            // 每一個字母
            for (const char& c : s)
            {
                if (p->m_next.count(c) <= 0) p->m_next.insert(make_pair(c, new TrieNode));
                p = p->m_next[c];
                if (p->m_isWord) break; // 剪枝,當已經有 prefix 存在
            }
            p->m_isWord = true;
        }

        string result;
        bool isFirst = true;
        for (const string& word : GetVec(sentence, ' '))
        {
            string tmp = GetByPrefix(root, word);
            if (isFirst)
                result += tmp;
            else
                result += " " + tmp;
            isFirst = false;
        }

        return result;
    }
private:
    vector<string> GetVec(const string& sentence, char delimiter)
    {
        string result;
        vector<string> res;
        istringstream iss(sentence);
        while (getline(iss, result, delimiter)) {
            res.push_back(result);
        }
        return res;
    }
    string GetByPrefix(const TrieNode& root, const string& word)
    {
        string result;
        TrieNode const * p = &root;
        for (const char& c : word)
        {
            result += c;
            auto it = p->m_next.find(c);
            if (it != p->m_next.end())
            {
                p = it->second;
                if (p->m_isWord) return result;
            }
            else
                return word;
        }

        return word;
    }
};
\end{Code}

\section{Design Search Autocomplete System} %%%%%%%%%%%%%%%%%%%%%%%%%%%%%%
\label{sec:design-search-autocomplete-system}


\subsubsection{描述}
Design a search autocomplete system for a search engine. Users may input a sentence (at least one word and end with a special character '\#'). For each character they type except '\#', you need to return the top 3 historical hot sentences that have prefix the same as the part of sentence already typed. Here are the specific rules:

\begindot
\item The hot degree for a sentence is defined as the number of times a user typed the exactly same sentence before.
\item The returned top 3 hot sentences should be sorted by hot degree (The first is the hottest one). If several sentences have the same degree of hot, you need to use ASCII-code order (smaller one appears first).
\item If less than 3 hot sentences exist, then just return as many as you can.
\item When the input is a special character, it means the sentence ends, and in this case, you need to return an empty list.
\myenddot

Your job is to implement the following functions:

The constructor function:

AutocompleteSystem(String[] sentences, int[] times): This is the constructor. The input is historical data. Sentences is a string array consists of previously typed sentences. Times is the corresponding times a sentence has been typed. Your system should record these historical data.

Now, the user wants to input a new sentence. The following function will provide the next character the user types:

List<String> input(char c): The input c is the next character typed by the user. The character will only be lower-case letters ('a' to 'z'), blank space (' ') or a special character ('\#'). Also, the previously typed sentence should be recorded in your system. The output will be the top 3 historical hot sentences that have prefix the same as the part of sentence already typed.

Example:
Operation: AutocompleteSystem(["i love you", "island","ironman", "i love leetcode"], [5,3,2,2])
The system have already tracked down the following sentences and their corresponding times:
"i love you" : 5 times
"island" : 3 times
"ironman" : 2 times
"i love leetcode" : 2 times
Now, the user begins another search:

Operation: input('i')
Output: ["i love you", "island","i love leetcode"]
Explanation:
There are four sentences that have prefix "i". Among them, "ironman" and "i love leetcode" have same hot degree. Since ' ' has ASCII code 32 and 'r' has ASCII code 114, "i love leetcode" should be in front of "ironman". Also we only need to output top 3 hot sentences, so "ironman" will be ignored.

Operation: input(' ')
Output: ["i love you","i love leetcode"]
Explanation:
There are only two sentences that have prefix "i ".

Operation: input('a')
Output: []
Explanation:
There are no sentences that have prefix "i a".

Operation: input('\#')
Output: []
Explanation:
The user finished the input, the sentence "i a" should be saved as a historical sentence in system. And the following input will be counted as a new search.

\subsubsection{暴力計算}
\begin{Code}
// 時間複雜度O(k * l),空間複雜度O(n + mlogm)
class AutocompleteSystem {
public:
    AutocompleteSystem(vector<string>& sentences, vector<int>& times) {
        if (sentences.size() == times.size())
        {
            // 記低每個字句的出現次數
            for (size_t i = 0; i < sentences.size(); i++)
                m_cache.emplace(sentences[i], times[i]);
        }
    }

    vector<string> input(char c) {
        if (c == '#')
        {
            m_cache[m_curSentence] += 1;
            m_curSentence = "";

            return vector<string>();
        }
        else
        {
            m_curSentence += c;
            vector<pair<string, int>> resultCount;
            // 歷遍每一個 sentence 找出 prefix
            for (const auto& [k, v] : m_cache)
            {
                // 找尋 prefix
                if (k.find(m_curSentence) == 0)
                {
                    resultCount.emplace_back(k, v);
                }
            }

            sort(resultCount.begin(), resultCount.end()
                 , [&](const auto& left, const auto& right)
                 {
                     // 先比較出現次數的多少
                     if (left.second == right.second)
                     {
                         // 後比較文字的先後
                         return left.first < right.first;
                     }
                     else
                         return left.second > right.second;
                 });

            vector<string> result; result.reserve(3);
            int loop = 0;
            for (const auto& [str, count] : resultCount)
            {
                result.push_back(str);
                if (++loop >= 3) break;
            }

            return result;
        }
    }
private:
    unordered_map<string, int> m_cache;
    string m_curSentence;
};
\end{Code}
