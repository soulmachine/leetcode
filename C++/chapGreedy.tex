\chapter{貪心法}


\section{Jump Game} %%%%%%%%%%%%%%%%%%%%%%%%%%%%%%
\label{sec:jump-game}


\subsubsection{描述}
Given an array of non-negative integers, you are initially positioned at the first index of the array.

Each element in the array represents your maximum jump length at that position.

Determine if you are able to reach the last index.

For example:

\code{A = [2,3,1,1,4]}, return true.

\code{A = [3,2,1,0,4]}, return false.


\subsubsection{分析}
由於每層最多可以跳\fn{A[i]}步,也可以跳0或1步,因此如果能到達最高層,則説明每一層都可以到達。有了這個條件,説明可以用貪心法。

思路一:正向,從0出發,一層一層網上跳,看最後能不能超過最高層,能超過,説明能到達,否則不能到達。

思路二:逆向,從最高層下樓梯,一層一層下降,看最後能不能下降到第0層。

思路三:如果不敢用貪心,可以用動規,設狀態為\fn{f[i]},表示從第0層出發,走到\fn{A[i]}時剩餘的最大步數,則狀態轉移方程為:
$$
f[i] = max(f[i-1], A[i-1])-1, i > 0
$$


\subsubsection{代碼1}
\begin{Code}
// LeetCode, Jump Game
// 思路1,時間複雜度O(n),空間複雜度O(1)
class Solution {
public:
    bool canJump(const vector<int>& nums) {
        int reach = 1; // 最右能跳到哪裏
        for (int i = 0; i < reach && reach < nums.size(); ++i)
            reach = max(reach,  i + 1 + nums[i]);
        return reach >= nums.size();
    }
};
\end{Code}


\subsubsection{代碼2}
\begin{Code}
// LeetCode, Jump Game
// 思路2,時間複雜度O(n),空間複雜度O(1)
class Solution {
public:
    bool canJump (const vector<int>& nums) {
        if (nums.empty()) return true;
        // 逆向下樓梯,最左能下降到第幾層
        int left_most = nums.size() - 1;

        for (int i = nums.size() - 2; i >= 0; --i)
            if (i + nums[i] >= left_most)
                left_most = i;

        return left_most == 0;
    }
};
\end{Code}


\subsubsection{代碼3}
\begin{Code}
// LeetCode, Jump Game
// 思路三,動規,時間複雜度O(n),空間複雜度O(n)
class Solution {
public:
    bool canJump(const vector<int>& nums) {
        vector<int> f(nums.size(), 0);
        f[0] = 0;
        for (int i = 1; i < nums.size(); i++) {
            f[i] = max(f[i - 1], nums[i - 1]) - 1;
            if (f[i] < 0) return false;;
        }
        return f[nums.size() - 1] >= 0;
    }
};
\end{Code}


\subsubsection{相關題目}
\begindot
\item Jump Game II ,見 \S \ref{sec:jump-game-ii}
\myenddot


\section{Jump Game II} %%%%%%%%%%%%%%%%%%%%%%%%%%%%%%
\label{sec:jump-game-ii}


\subsubsection{描述}
Given an array of non-negative integers, you are initially positioned at the first index of the array.

Each element in the array represents your maximum jump length at that position.

Your goal is to reach the last index in the minimum number of jumps.

For example:
Given array \code{A = [2,3,1,1,4]}

The minimum number of jumps to reach the last index is 2. (Jump 1 step from index 0 to 1, then 3 steps to the last index.)


\subsubsection{分析}
貪心法。


\subsubsection{代碼1}
\begin{Code}
// LeetCode, Jump Game II
// 時間複雜度O(n),空間複雜度O(1)
class Solution {
public:
    int jump(const vector<int>& nums) {
        int step = 0; // 最小步數
        int left = 0;
        int right = 0;  // [left, right]是當前能覆蓋的區間
        if (nums.size() == 1) return 0;

        while (left <= right) { // 嘗試從每一層跳最遠
            ++step;
            const int old_right = right;
            for (int i = left; i <= old_right; ++i) {
                int new_right = i + nums[i];
                if (new_right >= nums.size() - 1) return step;

                if (new_right > right) right = new_right;
            }
            left = old_right + 1;
        }
        return 0;
    }
};
\end{Code}


\subsubsection{代碼2}
\begin{Code}
// LeetCode, Jump Game II
// 時間複雜度O(n),空間複雜度O(1)
class Solution {
public:
    int jump(const vector<int>& nums) {
        int result = 0;
        // the maximum distance that has been reached
        int last = 0;
        // the maximum distance that can be reached by using "ret+1" steps
        int cur = 0;
        for (int i = 0; i < nums.size(); ++i) {
            if (i > last) {
                last = cur;
                ++result;
            }
            cur = max(cur, i + nums[i]);
        }

        return result;
    }
};
\end{Code}


\subsubsection{相關題目}
\begindot
\item Jump Game ,見 \S \ref{sec:jump-game}
\myenddot


\section{Best Time to Buy and Sell Stock} %%%%%%%%%%%%%%%%%%%%%%%%%%%%%%
\label{sec:best-time-to-buy-and-sell-stock}


\subsubsection{描述}
Say you have an array for which the i-th element is the price of a given stock on day i.

If you were only permitted to complete at    most one transaction (ie, buy one and sell one share of the stock), design an algorithm to find the maximum profit.


\subsubsection{分析}
貪心法,分別找到價格最低和最高的一天,低進高出,注意最低的一天要在最高的一天之前。

把原始價格序列變成差分序列,本題也可以做是最大$m$子段和,$m=1$。

\subsubsection{代碼}
\begin{Code}
// LeetCode, Best Time to Buy and Sell Stock
// 時間複雜度O(n),空間複雜度O(1)
class Solution {
public:
    int maxProfit(vector<int> &prices) {
        if (prices.size() < 2) return 0;
        int profit = 0;  // 差價,也就是利潤
        int cur_min = prices[0]; // 當前最小

        for (int i = 1; i < prices.size(); i++) {
            profit = max(profit, prices[i] - cur_min);
            cur_min = min(cur_min, prices[i]);
        }
        return profit;
    }
};
\end{Code}


\subsubsection{相關題目}
\begindot
\item Best Time to Buy and Sell Stock II,見 \S \ref{sec:best-time-to-buy-and-sell-stock-ii}
\item Best Time to Buy and Sell Stock III,見 \S \ref{sec:best-time-to-buy-and-sell-stock-iii}
\myenddot


\section{Best Time to Buy and Sell Stock II} %%%%%%%%%%%%%%%%%%%%%%%%%%%%%%
\label{sec:best-time-to-buy-and-sell-stock-ii}


\subsubsection{描述}
Say you have an array for which the i-th element is the price of a given stock on day i.

Design an algorithm to find the maximum profit. You may complete as many transactions as you like (ie, buy one and sell one share of the stock multiple times). However, you may not engage in multiple transactions at the same time (ie, you must sell the stock before you buy again).


\subsubsection{分析}
貪心法,低進高出,把所有正的價格差價相加起來。

把原始價格序列變成差分序列,本題也可以做是最大$m$子段和,$m=$數組長度。

\subsubsection{代碼}
\begin{Code}
// LeetCode, Best Time to Buy and Sell Stock II
// 時間複雜度O(n),空間複雜度O(1)
class Solution {
public:
    int maxProfit(vector<int> &prices) {
        int sum = 0;
        for (int i = 1; i < prices.size(); i++) {
            int diff = prices[i] - prices[i - 1];
            if (diff > 0) sum += diff;
        }
        return sum;
    }
};
\end{Code}


\subsubsection{相關題目}
\begindot
\item Best Time to Buy and Sell Stock,見 \S \ref{sec:best-time-to-buy-and-sell-stock}
\item Best Time to Buy and Sell Stock III,見 \S \ref{sec:best-time-to-buy-and-sell-stock-iii}
\myenddot


\section{Longest Substring Without Repeating Characters} %%%%%%%%%%%%%%%%%%%%%%%%%%%%%%
\label{sec:longest-substring-without-repeating-characters}


\subsubsection{描述}
Given a string, find the length of the longest substring without repeating characters. For example, the longest substring without repeating letters for \code{"abcabcbb"} is \code{"abc"}, which the length is 3. For \code{"bbbbb"} the longest substring is \code{"b"}, with the length of 1.


\subsubsection{分析}
假設子串裏含有重複字符,則父串一定含有重複字符,單個子問題就可以決定父問題,因此可以用貪心法。跟動規不同,動規裏,單個子問題只能影響父問題,不足以決定父問題。

從左往右掃描,當遇到重複字母時,以上一個重複字母的\fn{index+1},作為新的搜索起始位置,直到最後一個字母,複雜度是$O(n)$。如圖~\ref{fig:longest-substring-without-repeating-characters}所示。

\begin{center}
\includegraphics[width=300pt]{longest-substring-without-repeating-characters.png}\\
\figcaption{不含重複字符的最長子串}\label{fig:longest-substring-without-repeating-characters}
\end{center}


\subsubsection{代碼}
\begin{Code}
// LeetCode, Longest Substring Without Repeating Characters
// 時間複雜度O(n),空間複雜度O(1)
// 考慮非字母的情況
class Solution {
public:
    int lengthOfLongestSubstring(string s) {
        const int ASCII_MAX = 255;
        int last[ASCII_MAX]; // 記錄字符上次出現過的位置
        int start = 0; // 記錄當前子串的起始位置

        fill(last, last + ASCII_MAX, -1); // 0也是有效位置,因此初始化為-1
        int max_len = 0;
        for (int i = 0; i < s.size(); i++) {
            if (last[s[i]] >= start) {
                max_len = max(i - start, max_len);
                start = last[s[i]] + 1;
            }
            last[s[i]] = i;
        }
        return max((int)s.size() - start, max_len);  // 別忘了最後一次,例如"abcd"
    }
};
\end{Code}


\subsubsection{相關題目}
\begindot
\item 無
\myenddot


\section{Container With Most Water}
\label{sec:container-with-most-water}


\subsubsection{描述}
Given $n$ non-negative integers $a_1, a_2, ..., a_n$, where each represents a point at coordinate $(i, a_i)$. n vertical lines are drawn such that the two endpoints of line $i$ is at $(i, a_i)$ and $(i, 0)$. Find two lines, which together with x-axis forms a container, such that the container contains the most water.

Note: You may not slant the container.


\subsubsection{分析}
每個容器的面積,取決於最短的木板。


\subsubsection{代碼}
\begin{Code}
// LeetCode, Container With Most Water
// 時間複雜度O(n),空間複雜度O(1)
class Solution {
public:
    int maxArea(vector<int> &height) {
        int start = 0;
        int end = height.size() - 1;
        int result = INT_MIN;
        while (start < end) {
            int area = min(height[end], height[start]) * (end - start);
            result = max(result, area);
            if (height[start] <= height[end]) {
                start++;
            } else {
                end--;
            }
        }
        return result;
    }
};
\end{Code}


\subsubsection{相關題目}
\begindot
\item Trapping Rain Water, 見 \S \ref{sec:trapping-rain-water}
\item Largest Rectangle in Histogram, 見 \S \ref{sec:largest-rectangle-in-histogram}
\myenddot
