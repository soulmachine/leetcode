\chapter{棧和隊列}


\section{棧} %%%%%%%%%%%%%%%%%%%%%%%%%%%%%%


\subsection{Valid Parentheses} %%%%%%%%%%%%%%%%%%%%%%%%%%%%%%
\label{sec:valid-parentheses}


\subsubsection{描述}
Given a string containing just the characters \code{'(', ')', '\{', '\}', '['} and \code{']'}, determine if the input string is valid.

The brackets must close in the correct order, \code{"()"} and \code{"()[]{}"} are all valid but \code{"(]"} and \code{"([)]"} are not.


\subsubsection{分析}
無


\subsubsection{代碼}
\begin{Code}
// LeetCode, Valid Parentheses
// 時間複雜度O(n),空間複雜度O(n)
class Solution {
public:
    bool isValid (string const& s) {
        string left = "([{";
        string right = ")]}";
        stack<char> stk;

        for (auto c : s) {
            if (left.find(c) != string::npos) {
                stk.push (c);
            } else {
                if (stk.empty () || stk.top () != left[right.find (c)])
                    return false;
                else
                    stk.pop ();
            }
        }
        return stk.empty();
    }
};
\end{Code}


\subsubsection{相關題目}
\begindot
\item Generate Parentheses, 見 \S \ref{sec:generate-parentheses}
\item Longest Valid Parentheses, 見 \S \ref{sec:longest-valid-parentheses}
\myenddot


\subsection{Longest Valid Parentheses} %%%%%%%%%%%%%%%%%%%%%%%%%%%%%%
\label{sec:longest-valid-parentheses}


\subsubsection{描述}
Given a string containing just the characters \code{'('} and \code{')'}, find the length of the longest valid (well-formed) parentheses substring.

For \code{"(()"}, the longest valid parentheses substring is \code{"()"}, which has length = 2.

Another example is \code{")()())"}, where the longest valid parentheses substring is \code{"()()"}, which has length = 4.


\subsubsection{分析}
無


\subsubsection{使用棧}
\begin{Code}
// LeetCode, Longest Valid Parenthese
// 使用棧,時間複雜度O(n),空間複雜度O(n)
class Solution {
public:
    int longestValidParentheses(const string& s) {
        int max_len = 0, last = -1; // the position of the last ')'
        stack<int> lefts;  // keep track of the positions of non-matching '('s

        for (int i = 0; i < s.size(); ++i) {
            if (s[i] =='(') {
                lefts.push(i);
            } else {
                if (lefts.empty()) {
                    // no matching left
                    last = i;
                } else {
                    // find a matching pair
                    lefts.pop();
                    if (lefts.empty()) {
                        max_len = max(max_len, i-last);
                    } else {
                        max_len = max(max_len, i-lefts.top());
                    }
                }
            }
        }
        return max_len;
    }
};
\end{Code}

\subsubsection{Dynamic Programming, One Pass}
\begin{Code}
// LeetCode, Longest Valid Parenthese
// 時間複雜度O(n),空間複雜度O(n)
// @author 一隻傑森(http://weibo.com/wjson)
class Solution {
public:
    int longestValidParentheses(const string& s) {
        vector<int> f(s.size(), 0);
        int ret = 0;
        for (int i = s.size() - 2; i >= 0; --i) {
            int match = i + f[i + 1] + 1;
            // case: "((...))"
            if (s[i] == '(' && match < s.size() && s[match] == ')') {
                f[i] = f[i + 1] + 2;
                // if a valid sequence exist afterwards "((...))()"
                if (match + 1 < s.size()) f[i] += f[match + 1];
            }
            ret = max(ret, f[i]);
        }
        return ret;
    }
};
\end{Code}


\subsubsection{兩遍掃描}
\begin{Code}
// LeetCode, Longest Valid Parenthese
// 兩遍掃描,時間複雜度O(n),空間複雜度O(1)
// @author 曹鵬(http://weibo.com/cpcs)
class Solution {
public:
    int longestValidParentheses(const string& s) {
        int answer = 0, depth = 0, start = -1;
        for (int i = 0; i < s.size(); ++i) {
            if (s[i] == '(') {
                ++depth;
            } else {
                --depth;
                if (depth < 0) {
                    start = i;
                    depth = 0;
                } else if (depth == 0) {
                    answer = max(answer, i - start);
                }
            }
        }

        depth = 0;
        start = s.size();
        for (int i = s.size() - 1; i >= 0; --i) {
            if (s[i] == ')') {
                ++depth;
            } else {
                --depth;
                if (depth < 0) {
                    start = i;
                    depth = 0;
                } else if (depth == 0) {
                    answer = max(answer, start - i);
                }
            }
        }
        return answer;
    }
};
\end{Code}


\subsubsection{相關題目}
\begindot
\item Valid Parentheses, 見 \S \ref{sec:valid-parentheses}
\item Generate Parentheses, 見 \S \ref{sec:generate-parentheses}
\myenddot


\subsection{Largest Rectangle in Histogram} %%%%%%%%%%%%%%%%%%%%%%%%%%%%%%
\label{sec:largest-rectangle-in-histogram}


\subsubsection{描述}
Given $n$ non-negative integers representing the histogram's bar height where the width of each bar is 1, find the area of largest rectangle in the histogram.

\begin{center}
\includegraphics[width=120pt]{histogram.png}\\
\figcaption{Above is a histogram where width of each bar is 1, given height = \fn{[2,1,5,6,2,3]}.}\label{fig:histogram}
\end{center}

\begin{center}
\includegraphics[width=120pt]{histogram-area.png}\\
\figcaption{The largest rectangle is shown in the shaded area, which has area = 10 unit.}\label{fig:histogram-area}
\end{center}

For example,
Given height = \fn{[2,1,5,6,2,3]},
return 10.


\subsubsection{分析}
簡單的,類似於 Container With Most Water(\S \ref{sec:container-with-most-water}),對每個柱子,左右擴展,直到碰到比自己矮的,計算這個矩形的面積,用一個變量記錄最大的面積,複雜度$O(n^2)$,會超時。

如圖\S \ref{fig:histogram-area}所示,從左到右處理直方,當$i=4$時,小於當前棧頂(即直方3),對於直方3,無論後面還是前面的直方,都不可能得到比目前棧頂元素更高的高度了,處理掉直方3(計算從直方3到直方4之間的矩形的面積,然後從棧裏彈出);對於直方2也是如此;直到碰到比直方4更矮的直方1。

這就意味着,可以維護一個遞增的棧,每次比較棧頂與當前元素。如果當前元素大於棧頂元素,則入棧,否則合併現有棧,直至棧頂元素小於當前元素。結尾時入棧元素0,重複合併一次。


\subsubsection{代碼}
\begin{Code}
// LeetCode, Largest Rectangle in Histogram
// 時間複雜度O(n),空間複雜度O(n)
class Solution {
public:
    int largestRectangleArea(vector<int> &height) {
        stack<int> s;
        height.push_back(0);
        int result = 0;
        for (int i = 0; i < height.size(); ) {
            if (s.empty() || height[i] > height[s.top()])
                s.push(i++);
            else {
                int tmp = s.top();
                s.pop();
                result = max(result,
                        height[tmp] * (s.empty() ? i : i - s.top() - 1));
            }
        }
        return result;
    }
};
\end{Code}

\subsubsection{遞歸版}
\begin{Code}
// LeetCode, Largest Rectangle in Histogram
// 時間複雜度O(nlogn ~ n^2),空間複雜度O(n)
// 會超時
class Solution {
public:
    int largestRectangleArea(vector<int>& heights) {
        if (heights.size() == 0) return 0;

        return DFS(heights, 0, heights.size()-1);
    }
private:
    int DFS(vector<int>& heights, int first, int last) {
        if (first > last) return 0;

        int minIndex = first;
        for (int i = first; i <= last; i++) {
            if (heights[minIndex] > heights[i])
                minIndex = i;
        }

        return max(heights[minIndex] * (last - first + 1)
                   , max(DFS(heights, first, minIndex-1)
                         , DFS(heights, minIndex+1, last)));
    }
};
\end{Code}

\subsubsection{遞歸版 - segment tree}
\begin{Code}
// LeetCode, Largest Rectangle in Histogram
// 時間複雜度O(nlogn),空間複雜度O(n)
// 把尋找 min index 的時間減少為 log n
class SegTreeNode {
public:
    int start;
    int end;
    int min;
    SegTreeNode *left;
    SegTreeNode *right;
    SegTreeNode(int start, int end) {
        this->start = start;
        this->end = end;
        left = right = NULL;
    }
};

class Solution {
public:
    int largestRectangleArea(vector<int>& heights) {
        if (heights.size() == 0) return 0;
        // first build a segment tree
        SegTreeNode *root = buildSegmentTree(heights, 0, heights.size() - 1);
        // next calculate the maximum area recursively
        return calculateMax(heights, root, 0, heights.size() - 1);
    }

    int calculateMax(vector<int>& heights, SegTreeNode* root, int start, int end) {
        if (start > end) {
            return -1;
        }
        if (start == end) {
            return heights[start];
        }
        int minIndex = query(root, heights, start, end);
        int leftMax = calculateMax(heights, root, start, minIndex - 1);
        int rightMax = calculateMax(heights, root, minIndex + 1, end);
        int minMax = heights[minIndex] * (end - start + 1);
        return max( max(leftMax, rightMax), minMax );
    }

    SegTreeNode *buildSegmentTree(vector<int>& heights, int start, int end) {
        if (start > end) return NULL;
        SegTreeNode *root = new SegTreeNode(start, end);
        if (start == end) {
            root->min = start;
            return root;
        } else {
            int middle = (start + end) / 2;
            root->left = buildSegmentTree(heights, start, middle);
            root->right = buildSegmentTree(heights, middle + 1, end);
            root->min = heights[root->left->min] < heights[root->right->min]
                        ? root->left->min : root->right->min;
            return root;
        }
    }

    int query(SegTreeNode *root, vector<int>& heights, int start, int end) {
        if (root == NULL || end < root->start || start > root->end) return -1;
        if (start <= root->start && end >= root->end) {
            return root->min;
        }
        int leftMin = query(root->left, heights, start, end);
        int rightMin = query(root->right, heights, start, end);
        if (leftMin == -1) return rightMin;
        if (rightMin == -1) return leftMin;
        return heights[leftMin] < heights[rightMin] ? leftMin : rightMin;
    }
};
\end{Code}

\subsubsection{相關題目}
\begindot
\item Trapping Rain Water, 見 \S \ref{sec:trapping-rain-water}
\item Container With Most Water, 見 \S \ref{sec:container-with-most-water}
\myenddot


\subsection{Evaluate Reverse Polish Notation} %%%%%%%%%%%%%%%%%%%%%%%%%%%%%%
\label{sec:evaluate-reverse-polish-notation}


\subsubsection{描述}
Evaluate the value of an arithmetic expression in Reverse Polish Notation.

Valid operators are \fn{+, -, *, /}. Each operand may be an integer or another expression.

Some examples:
\begin{Code}
  ["2", "1", "+", "3", "*"] -> ((2 + 1) * 3) -> 9
  ["4", "13", "5", "/", "+"] -> (4 + (13 / 5)) -> 6
\end{Code}


\subsubsection{分析}
無


\subsubsection{遞歸版}
\begin{Code}
// LeetCode, Evaluate Reverse Polish Notation
// 遞歸,時間複雜度O(n),空間複雜度O(logn)
class Solution {
public:
    int evalRPN(vector<string> &tokens) {
        int x, y;
        auto token = tokens.back();  tokens.pop_back();
        if (is_operator(token))  {
            y = evalRPN(tokens);
            x = evalRPN(tokens);
            if (token[0] == '+')       x += y;
            else if (token[0] == '-')  x -= y;
            else if (token[0] == '*')  x *= y;
            else                       x /= y;
        } else  {
            size_t i;
            x = stoi(token, &i);
        }
        return x;
    }
private:
    bool is_operator(const string &op) {
        return op.size() == 1 && string("+-*/").find(op) != string::npos;
    }
};
\end{Code}


\subsubsection{迭代版}
\begin{Code}
// LeetCode, Max Points on a Line
// 迭代,時間複雜度O(n),空間複雜度O(logn)
class Solution {
public:
    int evalRPN(vector<string> &tokens) {
        stack<string> s;
        for (auto token : tokens) {
            if (!is_operator(token)) {
                s.push(token);
            } else {
                int y = stoi(s.top());
                s.pop();
                int x = stoi(s.top());
                s.pop();
                if (token[0] == '+')       x += y;
                else if (token[0] == '-')  x -= y;
                else if (token[0] == '*')  x *= y;
                else                       x /= y;
                s.push(to_string(x));
            }
        }
        return stoi(s.top());
    }
private:
    bool is_operator(const string &op) {
        return op.size() == 1 && string("+-*/").find(op) != string::npos;
    }
};
\end{Code}


\subsubsection{相關題目}
\begindot
\item 無
\myenddot


\section{隊列} %%%%%%%%%%%%%%%%%%%%%%%%%%%%%%

\subsection{Sliding Window Maximum} %%%%%%%%%%%%%%%%%%%%%%%%%%%%%%
\label{sec:sliding-window-maximum}


\subsubsection{描述}
Given an array nums, there is a sliding window of size k which is moving from the very left of the array to the very right. You can only see the k numbers in the window. Each time the sliding window moves right by one position. Return the max sliding window.


\textbf{Follow up:}
Could you solve it in linear time?

\textbf{Example:}
\begin{Code}
Input: nums = [1,3,-1,-3,5,3,6,7], and k = 3
Output: [3,3,5,5,6,7]
Explanation:

Window position                Max
---------------               -----
[1  3  -1] -3  5  3  6  7       3
 1 [3  -1  -3] 5  3  6  7       3
 1  3 [-1  -3  5] 3  6  7       5
 1  3  -1 [-3  5  3] 6  7       5
 1  3  -1  -3 [5  3  6] 7       6
 1  3  -1  -3  5 [3  6  7]      7
\end{Code}


\subsubsection{分析}
以 deque 來追踪視窗內的指標, 移除老去的指標, 移除數值太細的元素指標


\subsubsection{Code}
\begin{Code}
// LeetCode
// 迭代,時間複雜度O(n),空間複雜度O(n)
class Solution {
public:
    vector<int> maxSlidingWindow(vector<int>& nums, int k) {
        int N = nums.size();
        if (N < k) return vector<int>();

        vector<int> result; result.reserve(N-k+1);
        deque<int> indexList;

        // 預準首 K 項
        for (size_t i = 0; i < k; i++)
        {
            while (!indexList.empty() && nums[indexList.back()] < nums[i])
                indexList.pop_back();
            indexList.push_back(i);
        }
        result.push_back(nums[indexList.front()]);

        // 完成剩下的元素
        for (size_t i = k; i < nums.size(); i++)
        {
            // 移除老去的元素
            while (!indexList.empty() && indexList.front() < i-k+1)
                indexList.pop_front();

            // 移除數值太細的元素
            while (!indexList.empty() && nums[indexList.back()] < nums[i])
                indexList.pop_back();
            indexList.push_back(i);

            result.push_back(nums[indexList.front()]);
        }

        return result;
    }
};
\end{Code}


\subsubsection{相關題目}
\begindot
\item 無
\myenddot

\subsection{Meeting Rooms II} %%%%%%%%%%%%%%%%%%%%%%%%%%%%%%
\label{sec:meeting-rooms-ii}


\subsubsection{描述}
Given an array of meeting time intervals consisting of start and end times [[s1,e1],[s2,e2],...] (si < ei), find the minimum number of conference rooms required.

\textbf{Example 1}
\begin{Code}
Input: [[0, 30],[5, 10],[15, 20]]
Output: 2
\end{Code}

\textbf{Example 2}
\begin{Code}
Input: [[7,10],[2,4]]
Output: 1
\end{Code}


\subsubsection{分析}
Nil

\subsubsection{Priority Queue}
\begin{Code}
// LeetCode
// 迭代,時間複雜度O(nlogn),空間複雜度O(n)
class Solution {
public:
    int minMeetingRooms(vector<vector<int>>& intervals) {
        // 順序所以時間段落
        sort(intervals.begin(), intervals.end()
             , [&](const auto& first, const auto& second)
               {
                   return first[0] < second[0];
               });
        // 利用 priority queue 找出最快完結的時間段落
        auto Compare = [](const auto& first, const auto& second)
                          { return first.second > second.second; };
        priority_queue<pair<int, int>, vector<pair<int, int>>, decltype(Compare)>
                          pq(Compare);

        int maxUsed = 0;
        for (const auto& interval : intervals)
        {
            // 移除用完的房間
            while (!pq.empty() && interval[0] >= pq.top().second)
                pq.pop();
            // 加入新的時間段落
            pq.push(make_pair(interval[0], interval[1]));
            // 更新同一時間的最大房間用量
            maxUsed = max(maxUsed, (int)pq.size());
        }

        return maxUsed;
    }
};
\end{Code}

\subsubsection{Start End Vector}
\begin{Code}
// LeetCode
// 迭代,時間複雜度O(nlogn),空間複雜度O(n)
class Solution {
public:
    int minMeetingRooms(vector<vector<int>>& intervals) {
        // 使用兩支 vector 一支為 startTime, 一支為 endTime
        vector<int> startTime, endTime;
        startTime.reserve(intervals.size());
        endTime.reserve(intervals.size());
        for (const auto& interval : intervals)
        {
            startTime.push_back(interval[0]);
            endTime.push_back(interval[1]);
        }
        // 由細至大
        sort(startTime.begin(), startTime.end());
        sort(endTime.begin(), endTime.end());
        // 計算房間多少
        int curRm = 0;
        int maxRm = 0;
        int ei = 0;
        for (int si = 0; si < (int)startTime.size(); si++)
        {
            while (startTime[si] >= endTime[ei])
            {
                ei++;
                curRm--;
            }
            curRm++;
            maxRm = max(maxRm, curRm);
        }

        return maxRm;
    }
};
\end{Code}

\subsubsection{相關題目}
\begindot
\item 無
\myenddot
