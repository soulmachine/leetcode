\chapter{廣度優先搜索}
當題目看不出任何規律,既不能用分治,貪心,也不能用動規時,這時候萬能方法——搜索,
就派上用場了。搜索分為廣搜和深搜,廣搜裏面又有普通廣搜,雙向廣搜,A*搜索等。
深搜裏面又有普通深搜,回溯法等。

廣搜和深搜非常類似(除了在擴展節點這部分不一樣),二者有相同的框架,如何表示狀態?
如何擴展狀態?如何判重?尤其是判重,解決了這個問題,基本上整個問題就解決了。
\newline


\section{Word Ladder} %%%%%%%%%%%%%%%%%%%%%%%%%%%%%%
\label{sec:word-ladder}


\subsubsection{描述}
Given two words (start and end), and a dictionary, find the length of shortest transformation sequence from start to end, such that:
\begindot
\item Only one letter can be changed at a time
\item Each intermediate word must exist in the dictionary
\myenddot

For example, Given:

\begin{Code}
start = "hit"
end = "cog"
dict = ["hot","dot","dog","lot","log","cog"]
\end{Code}
As one shortest transformation is \code{"hit" -> "hot" -> "dot" -> "dog" -> "cog"}, return its length $5$.

Note:
\begindot
\item Return 0 if there is no such transformation sequence.
\item All words have the same length.
\item All words contain only lowercase alphabetic characters.
\myenddot


\subsubsection{分析}
求最短路徑,用廣搜。


\subsubsection{單隊列}
\begin{Code}
//LeetCode, Word Ladder
// 時間複雜度O(n),空間複雜度O(n)
struct state_t {
    string word;
    int level;

    state_t() { word = ""; level = 0; }
    state_t(const string& word, int level) {
        this->word = word;
        this->level = level;
    }

    bool operator==(const state_t &other) const {
        return this->word == other.word;
    }
};

namespace std {
    template<> struct hash<state_t> {
    public:
        size_t operator()(const state_t& s) const {
            return str_hash(s.word);
        }
    private:
        std::hash<std::string> str_hash;
    };
}


class Solution {
public:
    int ladderLength(const string& start, const string &end,
            const unordered_set<string> &dict) {
        queue<state_t> q;
        unordered_set<state_t> visited;  // 判重

        auto state_is_valid = [&](const state_t& s) {
            return dict.find(s.word) != dict.end();
        };
        auto state_is_target = [&](const state_t &s) {return s.word == end; };
        auto state_extend = [&](const state_t &s) {
            unordered_set<state_t> result;

            for (size_t i = 0; i < s.word.size(); ++i) {
                state_t new_state(s.word, s.level + 1);
                for (char c = 'a'; c <= 'z'; c++) {
                    // 防止同字母替換
                    if (c == new_state.word[i]) continue;

                    swap(c, new_state.word[i]);

                    if (state_is_valid(new_state) &&
                        visited.find(new_state) == visited.end()) {
                        result.insert(new_state);
                    }
                    swap(c, new_state.word[i]); // 恢復該單詞
                }
            }

            return result;
        };

        state_t start_state(start, 0);
        q.push(start_state);
        visited.insert(start_state);
        while (!q.empty()) {
            // 千萬不能用 const auto&,pop() 會刪除元素,
            // 引用就變成了懸空引用
            const auto state = q.front();
            q.pop();

            if (state_is_target(state)) {
                return state.level + 1;
            }

            const auto& new_states = state_extend(state);
            for (const auto& new_state : new_states) {
                q.push(new_state);
                visited.insert(new_state);
            }
        }
        return 0;
    }
};
\end{Code}


\subsubsection{雙隊列}
\begin{Code}
//LeetCode, Word Ladder
// 時間複雜度O(n),空間複雜度O(n)
class Solution {
public:
    int ladderLength(const string& start, const string &end,
            const unordered_set<string> &dict) {
        queue<string> current, next;    // 當前層,下一層
        unordered_set<string> visited;  // 判重

        int level = -1;  // 層次

        auto state_is_valid = [&](const string& s) {
            return dict.find(s) != dict.end();
        };
        auto state_is_target = [&](const string &s) {return s == end;};
        auto state_extend = [&](const string &s) {
            unordered_set<string> result;

            for (size_t i = 0; i < s.size(); ++i) {
                string new_word(s);
                for (char c = 'a'; c <= 'z'; c++) {
                    // 防止同字母替換
                    if (c == new_word[i]) continue;

                    swap(c, new_word[i]);

                    if (state_is_valid(new_word) &&
                        visited.find(new_word) == visited.end()) {
                        result.insert(new_word);
                    }
                    swap(c, new_word[i]); // 恢復該單詞
                }
            }

            return result;
        };

        current.push(start);
        visited.insert(start);
        while (!current.empty()) {
            ++level;
            while (!current.empty()) {
                // 千萬不能用 const auto&,pop() 會刪除元素,
                // 引用就變成了懸空引用
                const auto state = current.front();
                current.pop();

                if (state_is_target(state)) {
                    return level + 1;
                }

                const auto& new_states = state_extend(state);
                for (const auto& new_state : new_states) {
                    next.push(new_state);
                    visited.insert(new_state);
                }
            }
            swap(next, current);
        }
        return 0;
    }
};
\end{Code}


\subsubsection{相關題目}

\begindot
\item Word Ladder II,見 \S \ref{sec:word-ladder-ii}
\myenddot


\section{Word Ladder II} %%%%%%%%%%%%%%%%%%%%%%%%%%%%%%
\label{sec:word-ladder-ii}


\subsubsection{描述}
Given two words (start and end), and a dictionary, find all shortest transformation sequence(s) from start to end, such that:
\begindot
\item Only one letter can be changed at a time
\item Each intermediate word must exist in the dictionary
\myenddot

For example, Given:
\begin{Code}
start = "hit"
end = "cog"
dict = ["hot","dot","dog","lot","log"]
\end{Code}
Return
\begin{Code}
[
    ["hit","hot","dot","dog","cog"],
    ["hit","hot","lot","log","cog"]
]
\end{Code}

Note:
\begindot
\item All words have the same length.
\item All words contain only lowercase alphabetic characters.
\myenddot


\subsubsection{分析}
跟 Word Ladder比,這題是求路徑本身,不是路徑長度,也是BFS,略微麻煩點。

求一條路徑和求所有路徑有很大的不同,求一條路徑,每個狀態節點只需要記錄一個前驅即可;求所有路徑時,有的狀態節點可能有多個父節點,即要記錄多個前驅。

如果當前路徑長度已經超過當前最短路徑長度,可以中止對該路徑的處理,因為我們要找的是最短路徑。


\subsubsection{單隊列}

\begin{Code}
//LeetCode, Word Ladder II
// 時間複雜度O(n),空間複雜度O(n)
struct state_t {
    string word;
    int level;

    state_t() { word = ""; level = 0; }
    state_t(const string& word, int level) {
        this->word = word;
        this->level = level;
    }

    bool operator==(const state_t &other) const {
        return this->word == other.word;
    }
};

namespace std {
    template<> struct hash<state_t> {
    public:
        size_t operator()(const state_t& s) const {
            return str_hash(s.word);
        }
    private:
        std::hash<std::string> str_hash;
    };
}


class Solution {
public:
    vector<vector<string> > findLadders(const string& start,
        const string& end, const unordered_set<string> &dict) {
        queue<state_t> q;
        unordered_set<state_t> visited; // 判重
        unordered_map<state_t, vector<state_t> > father; // DAG (樹) key: child value: father

        auto state_is_valid = [&](const state_t& s) {
            return dict.find(s.word) != dict.end();
        };
        auto state_is_target = [&](const state_t &s) {return s.word == end; };
        auto state_extend = [&](const state_t &s) {
            unordered_set<state_t> result;

            for (size_t i = 0; i < s.word.size(); ++i) {
                state_t new_state(s.word, s.level + 1);
                for (char c = 'a'; c <= 'z'; c++) {
                    // 防止同字母替換
                    if (c == new_state.word[i]) continue;

                    swap(c, new_state.word[i]);

                    if (state_is_valid(new_state)) {
                        auto visited_iter = visited.find(new_state);

                        if (visited_iter != visited.end()) {
                            if (visited_iter->level < new_state.level) {
                                // do nothing
                            } else if (visited_iter->level == new_state.level) {
                                result.insert(new_state);
                            } else { // not possible
                                throw std::logic_error("not possible to get here");
                            }
                        } else {
                            result.insert(new_state);
                        }
                    }
                    swap(c, new_state.word[i]); // 恢復該單詞
                }
            }

            return result;
        };

        vector<vector<string>> result;
        state_t start_state(start, 0);
        q.push(start_state);
        visited.insert(start_state);
        while (!q.empty()) {
            // 千萬不能用 const auto&,pop() 會刪除元素,
            // 引用就變成了懸空引用
            const auto state = q.front();
            q.pop();

            // 如果當前路徑長度已經超過當前最短路徑長度,
            // 可以中止對該路徑的處理,因為我們要找的是最短路徑
            if (!result.empty() && state.level + 1 > result[0].size()) break;

            if (state_is_target(state)) {
                vector<string> path;
                gen_path(father, start_state, state, path, result);
                continue;
            }
            // 必須挪到下面,比如同一層A和B兩個節點均指向了目標節點,
            // 那麼目標節點就會在q中出現兩次,輸出路徑就會翻倍
            // visited.insert(state);

            // 擴展節點
            const auto& new_states = state_extend(state);
            for (const auto& new_state : new_states) {
                if (visited.find(new_state) == visited.end()) {
                    q.push(new_state);
                }
                visited.insert(new_state);
                father[new_state].push_back(state);
            }
        }

        return result;
    }
private:
    void gen_path(unordered_map<state_t, vector<state_t> > &father,
        const state_t &start, const state_t &state, vector<string> &path,
        vector<vector<string> > &result) {
        path.push_back(state.word);
        if (state == start) {
            if (!result.empty()) {
                if (path.size() < result[0].size()) {
                    result.clear();
                    result.push_back(path);
                    reverse(result.back().begin(), result.back().end());
                } else if (path.size() == result[0].size()) {
                    result.push_back(path);
                    reverse(result.back().begin(), result.back().end());
                } else { // not possible
                    throw std::logic_error("not possible to get here ");
                }
            } else {
                result.push_back(path);
                reverse(result.back().begin(), result.back().end());
            }

        } else {
            for (const auto& f : father[state]) {
                gen_path(father, start, f, path, result);
            }
        }
        path.pop_back();
    }
};
\end{Code}


\subsubsection{雙隊列}

\begin{Code}
//LeetCode, Word Ladder II
// 時間複雜度O(n),空間複雜度O(n)
class Solution {
public:
    vector<vector<string> > findLadders(const string& start,
            const string& end, const unordered_set<string> &dict) {
        // 當前層,下一層,用unordered_set是為了去重,例如兩個父節點指向
        // 同一個子節點,如果用vector, 子節點就會在next裏出現兩次,其實此
        // 時 father 已經記錄了兩個父節點,next裏重複出現兩次是沒必要的
        unordered_set<string> current, next;
        unordered_set<string> visited; // 判重
        unordered_map<string, vector<string> > father; // DAG

        int level = -1;  // 層次

        auto state_is_valid = [&](const string& s) {
            return dict.find(s) != dict.end();
        };
        auto state_is_target = [&](const string &s) {return s == end;};
        auto state_extend = [&](const string &s) {
            unordered_set<string> result;

            for (size_t i = 0; i < s.size(); ++i) {
                string new_word(s);
                for (char c = 'a'; c <= 'z'; c++) {
                    // 防止同字母替換
                    if (c == new_word[i]) continue;

                    swap(c, new_word[i]);

                    if (state_is_valid(new_word) &&
                            visited.find(new_word) == visited.end()) {
                        result.insert(new_word);
                    }
                    swap(c, new_word[i]); // 恢復該單詞
                }
            }

            return result;
        };

        vector<vector<string> > result;
        current.insert(start);
        while (!current.empty()) {
            ++ level;
            // 如果當前路徑長度已經超過當前最短路徑長度,可以中止對該路徑的
            // 處理,因為我們要找的是最短路徑
            if (!result.empty() && level+1 > result[0].size()) break;

            // 1. 延遲加入visited, 這樣才能允許兩個父節點指向同一個子節點
            // 2. 一股腦current 全部加入visited, 是防止本層前一個節點擴展
            // 節點時,指向了本層後面尚未處理的節點,這條路徑必然不是最短的
            for (const auto& state : current)
                visited.insert(state);
            for (const auto& state : current) {
                if (state_is_target(state)) {
                    vector<string> path;
                    gen_path(father, path, start, state, result);
                    continue;
                }

                const auto new_states = state_extend(state);
                for (const auto& new_state : new_states) {
                    next.insert(new_state);
                    father[new_state].push_back(state);
                }
            }

            current.clear();
            swap(current, next);
        }

        return result;
    }
private:
    void gen_path(unordered_map<string, vector<string> > &father,
            vector<string> &path, const string &start, const string &word,
            vector<vector<string> > &result) {
        path.push_back(word);
        if (word == start) {
            if (!result.empty()) {
                if (path.size() < result[0].size()) {
                    result.clear();
                    result.push_back(path);
                } else if(path.size() == result[0].size()) {
                    result.push_back(path);
                } else {
                    // not possible
                    throw std::logic_error("not possible to get here");
                }
            } else {
                result.push_back(path);
            }
            reverse(result.back().begin(), result.back().end());
        } else {
            for (const auto& f : father[word]) {
                gen_path(father, path, start, f, result);
            }
        }
        path.pop_back();
    }
};
\end{Code}


\subsubsection{圖的廣搜}

本題還可以看做是圖上的廣搜。給定了字典 \fn{dict},可以基於它畫出一個無向圖,表示單詞之間可以互相轉換。本題的本質就是已知起點和終點,在圖上找出所有最短路徑。

\begin{Code}
//LeetCode, Word Ladder II
// 時間複雜度O(n),空間複雜度O(n)
class Solution {
public:
    vector<vector<string> > findLadders(const string& start,
            const string &end, const unordered_set<string> &dict) {
        const auto& g = build_graph(dict);
        vector<state_t*> pool;
        queue<state_t*> q; // 未處理的節點
        // value 是所在層次
        unordered_map<string, int> visited;

        auto state_is_target = [&](const state_t *s) {return s->word == end; };

        vector<vector<string>> result;
        q.push(create_state(nullptr, start, 0, pool));
        while (!q.empty()) {
            state_t* state = q.front();
            q.pop();

            // 如果當前路徑長度已經超過當前最短路徑長度,
            // 可以中止對該路徑的處理,因為我們要找的是最短路徑
            if (!result.empty() && state->level+1 > result[0].size()) break;

            if (state_is_target(state)) {
                const auto& path = gen_path(state);
                if (result.empty()) {
                    result.push_back(path);
                } else {
                    if (path.size() < result[0].size()) {
                        result.clear();
                        result.push_back(path);
                    } else if (path.size() == result[0].size()) {
                        result.push_back(path);
                    } else {
                        // not possible
                        throw std::logic_error("not possible to get here");
                    }
                }
                continue;
            }

            // 擴展節點
            auto iter = g.find(state->word);
            if (iter == g.end()) continue;

            for (const auto& neighbor : iter->second) {
                auto visited_iter = visited.find(neighbor);

                if (visited_iter != visited.end() && 
                    visited_iter->second < state->level + 1) {
                    continue;
                }

                visited[neighbor] = state->level;
                q.push(create_state(state, neighbor, state->level + 1, pool));
            }
        }

        // release all states
        for (auto state : pool) {
            delete state;
        }
        return result;
    }

private:
    struct state_t {
        state_t* father;
        string word;
        int level; // 所在層次,從0開始編號

        state_t(state_t* father_, const string& word_, int level_) :
            father(father_), word(word_), level(level_) {}
    };

    state_t* create_state(state_t* parent, const string& value,
            int length, vector<state_t*>& pool) {
        state_t* node = new state_t(parent, value, length);
        pool.push_back(node);

        return node;
    }
    vector<string> gen_path(const state_t* node) {
        vector<string> path;

        while(node != nullptr) {
            path.push_back(node->word);
            node = node->father;
        }

        reverse(path.begin(), path.end());
        return path;
    }

    unordered_map<string, unordered_set<string> > build_graph(
            const unordered_set<string>& dict) {
        unordered_map<string, unordered_set<string> > adjacency_list;

        for (const auto& word : dict) {
            for (size_t i = 0; i < word.size(); ++i) {
                string new_word(word);
                for (char c = 'a'; c <= 'z'; c++) {
                    // 防止同字母替換
                    if (c == new_word[i]) continue;

                    swap(c, new_word[i]);

                    if ((dict.find(new_word) != dict.end())) {
                        auto iter = adjacency_list.find(word);
                        if (iter != adjacency_list.end()) {
                            iter->second.insert(new_word);
                        } else {
                            adjacency_list.insert(pair<string,
                                unordered_set<string>>(word, unordered_set<string>()));
                            adjacency_list[word].insert(new_word);
                        }
                    }
                    swap(c, new_word[i]); // 恢復該單詞
                }
            }
        }
        return adjacency_list;
    }
};
\end{Code}


\subsubsection{相關題目}

\begindot
\item Word Ladder,見 \S \ref{sec:word-ladder}
\myenddot


\section{Surrounded Regions} %%%%%%%%%%%%%%%%%%%%%%%%%%%%%%
\label{sec:surrounded-regions}


\subsubsection{描述}
Given a 2D board containing \fn{'X'} and \fn{'O'}, capture all regions surrounded by \fn{'X'}.

A region is captured by flipping all \fn{'O'}s into \fn{'X'}s in that surrounded region .

For example,
\begin{Code}
X X X X
X O O X
X X O X
X O X X
\end{Code}

After running your function, the board should be:
\begin{Code}
X X X X
X X X X
X X X X
X O X X
\end{Code}


\subsubsection{分析}
廣搜。從上下左右四個邊界往裏走,凡是能碰到的\fn{'O'},都是跟邊界接壤的,應該保留。


\subsubsection{代碼}
\begin{Code}
// LeetCode, Surrounded Regions
// BFS,時間複雜度O(n),空間複雜度O(n)
class Solution {
public:
    void solve(vector<vector<char>> &board) {
        if (board.empty()) return;

        const int m = board.size();
        const int n = board[0].size();
        for (int i = 0; i < n; i++) {
            bfs(board, 0, i);
            bfs(board, m - 1, i);
        }
        for (int j = 1; j < m - 1; j++) {
            bfs(board, j, 0);
            bfs(board, j, n - 1);
        }
        for (int i = 0; i < m; i++)
            for (int j = 0; j < n; j++)
                if (board[i][j] == 'O')
                    board[i][j] = 'X';
                else if (board[i][j] == '+')
                    board[i][j] = 'O';
    }
private:
    void bfs(vector<vector<char>> &board, int i, int j) {
        typedef pair<int, int> state_t;
        queue<state_t> q;
        const int m = board.size();
        const int n = board[0].size();

        auto state_is_valid = [&](const state_t &s) {
            const int x = s.first;
            const int y = s.second;
            if (x < 0 || x >= m || y < 0 || y >= n || board[x][y] != 'O')
                return false;
            return true;
        };

        auto state_extend = [&](const state_t &s) {
            vector<state_t> result;
            const int x = s.first;
            const int y = s.second;
            // 上下左右
            const state_t new_states[4] = {{x-1,y}, {x+1,y},
                    {x,y-1}, {x,y+1}};
            for (int k = 0; k < 4;  ++k) {
                if (state_is_valid(new_states[k])) {
                    // 既有標記功能又有去重功能
                    board[new_states[k].first][new_states[k].second] = '+';
                    result.push_back(new_states[k]);
                }
            }

            return result;
        };

        state_t start = { i, j };
        if (state_is_valid(start)) {
            board[i][j] = '+';
            q.push(start);
        }
        while (!q.empty()) {
            auto cur = q.front();
            q.pop();
            auto new_states = state_extend(cur);
            for (auto s : new_states) q.push(s);
        }
    }
};
\end{Code}


\subsubsection{相關題目}

\begindot
\item 無
\myenddot

\section{Walls and Gates}
\label{sec:walls-and-gates}

\subsubsection{描述}
You are given a m x n 2D grid initialized with these three possible values.

\begindot
\item -1 - A wall or an obstacle.
\item 0 - A gate.
\item INF - Infinity means an empty room. We use the value 231 - 1 = 2147483647 to represent INF as you may assume that the distance to a gate is less than 2147483647.
\myenddot

Fill each empty room with the distance to its nearest gate. If it is impossible to reach a gate, it should be filled with INF.

Example:
Given the 2D grid:
\begin{Code}
INF  -1  0  INF
INF INF INF  -1
INF  -1 INF  -1
  0  -1 INF INF
\end{Code}

After running your function, the 2D grid should be:
\begin{Code}
  3  -1   0   1
  2   2   1  -1
  1  -1   2  -1
  0  -1   3   4
\end{Code}


\subsubsection{BFS}
\begin{Code}
// LeetCode
// 時間複雜度O(n*m),空間複雜度O(1)
int rows[] = {0, 0, 1,-1};
int cols[] = {1,-1, 0, 0};
#define INF 2147483647
class Solution {
public:
    bool isSafe(vector<vector<int>>& rooms,int i, int j)
    {
        if( i <0 or i>=rooms.size() or j <0 or j >= rooms[0].size() or rooms[i][j] != INF )
            return false;
        return true;
    }

    void wallsAndGates(vector<vector<int>>& rooms) {
        int r = rooms.size();
        if (r < 1) return;
        int c = rooms[0].size();
        queue<pair<int,int>> q;
        for (int i=0; i < r; i++)
            for(int j=0; j <c; j++)
                if (rooms[i][j]==0)
                    q.push({i, j}); // 由每一個閘口開始

        while(!q.empty())
        {
            auto p = q.front(); q.pop();

            for(int d=0; d<4; d++)
            {
                int x = p.first  + rows[d];
                int y = p.second + cols[d];
                if (isSafe(rooms,x,y) )
                {
                    rooms[x][y] =  rooms[p.first][p.second] + 1;
                    q.push({x,y});
                }
            }
        }
    }
};
\end{Code}

\section{小結} %%%%%%%%%%%%%%%%%%%%%%%%%%%%%%
\label{sec:bfs-template}


\subsection{適用場景}

\textbf{輸入數據}:沒什麼特徵,不像深搜,需要有“遞歸”的性質。如果是樹或者圖,概率更大。

\textbf{狀態轉換圖}:樹或者DAG圖。

\textbf{求解目標}:求最短。


\subsection{思考的步驟}
\begin{enumerate}
\item 是求路徑長度,還是路徑本身(或動作序列)?
    \begin{enumerate}
    \item 如果是求路徑長度,則狀態裏面要存路徑長度(或雙隊列+一個全局變量)
    \item 如果是求路徑本身或動作序列
        \begin{enumerate}
            \item 要用一棵樹存儲寬搜過程中的路徑
            \item 是否可以預估狀態個數的上限?能夠預估狀態總數,則開一個大數組,用樹的雙親表示法;如果不能預估狀態總數,則要使用一棵通用的樹。這一步也是第4步的必要不充分條件。
        \end{enumerate}
    \end{enumerate}

\item 如何表示狀態?即一個狀態需要存儲哪些些必要的數據,才能夠完整提供如何擴展到下一步狀態的所有信息。一般記錄當前位置或整體局面。

\item 如何擴展狀態?這一步跟第2步相關。狀態裏記錄的數據不同,擴展方法就不同。對於固定不變的數據結構(一般題目直接給出,作為輸入數據),如二叉樹,圖等,擴展方法很簡單,直接往下一層走,對於隱式圖,要先在第1步裏想清楚狀態所帶的數據,想清楚了這點,那如何擴展就很簡單了。

\item 如何判斷重複?如果狀態轉換圖是一顆樹,則永遠不會出現迴路,不需要判重;如果狀態轉換圖是一個圖(這時候是一個圖上的BFS),則需要判重。
    \begin{enumerate}
    \item 如果是求最短路徑長度或一條路徑,則只需要讓“點”(即狀態)不重複出現,即可保證不出現迴路
    \item 如果是求所有路徑,注意此時,狀態轉換圖是DAG,即允許兩個父節點指向同一個子節點。具體實現時,每個節點要\textbf{“延遲”}加入到已訪問集合\fn{visited},要等一層全部訪問完後,再加入到\fn{visited}集合。
    \item 具體實現
        \begin{enumerate}
        \item 狀態是否存在完美哈希方案?即將狀態一一映射到整數,互相之間不會衝突。
        \item 如果不存在,則需要使用通用的哈希表(自己實現或用標準庫,例如\fn{unordered_set})來判重;自己實現哈希表的話,如果能夠預估狀態個數的上限,則可以開兩個數組,head和next,表示哈希表,參考第 \S \ref{subsec:eightDigits}節方案2。
        \item 如果存在,則可以開一個大布爾數組,來判重,且此時可以精確計算出狀態總數,而不僅僅是預估上限。
        \end{enumerate}
    \end{enumerate}

\item 目標狀態是否已知?如果題目已經給出了目標狀態,可以帶來很大便利,這時候可以從起始狀態出發,正向廣搜;也可以從目標狀態出發,逆向廣搜;也可以同時出發,雙向廣搜。
\end{enumerate}


\subsection{代碼模板}
廣搜需要一個隊列,用於一層一層擴展,一個hashset,用於判重,一棵樹(只求長度時不需要),用於存儲整棵樹。

對於隊列,可以用\fn{queue},也可以把\fn{vector}當做隊列使用。當求長度時,有兩種做法:
\begin{enumerate}
\item 只用一個隊列,但在狀態結構體\fn{state_t}裏增加一個整數字段\fn{level},表示當前所在的層次,當碰到目標狀態,直接輸出\fn{level}即可。這個方案,可以很容易的變成A*算法,把\fn{queue}替換為\fn{priority_queue}即可。
\item 用兩個隊列,\fn{current, next},分別表示當前層次和下一層,另設一個全局整數\fn{level},表示層數(也即路徑長度),當碰到目標狀態,輸出\fn{level}即可。這個方案,狀態裏可以不存路徑長度,只需全局設置一個整數\fn{level},比較節省內存;
\end{enumerate}

對於hashset,如果有完美哈希方案,用布爾數組(\fn{bool visited[STATE_MAX]}或\fn{vector<bool> visited(STATE_MAX, false)})來表示;如果沒有,可以用STL裏的\fn{set}或\fn{unordered_set}。

對於樹,如果用STL,可以用\fn{unordered_map<state_t, state_t > father}表示一顆樹,代碼非常簡潔。如果能夠預估狀態總數的上限(設為STATE_MAX),可以用數組(\fn{state_t nodes[STATE_MAX]}),即樹的雙親表示法來表示樹,效率更高,當然,需要寫更多代碼。


\subsubsection{如何表示狀態}

\begin{Codex}[label=bfs_common.h]
/** 狀態 */
struct state_t {
    int data1;  /** 狀態的數據,可以有多個字段. */
    int data2;  /** 狀態的數據,可以有多個字段. */
    // dataN;   /** 其他字段 */
    int action; /** 由父狀態移動到本狀態的動作,求動作序列時需要. */
    int level;  /** 所在的層次(從0開始),也即路徑長度-1,求路徑長度時需要;
                    不過,採用雙隊列時不需要本字段,只需全局設一個整數 */
    bool operator==(const state_t &other) const {
        return true;  // 根據具體問題實現
    }
};

// 定義hash函數

// 方法1:模板特化,當hash函數只需要狀態本身,不需要其他數據時,用這個方法比較簡潔
namespace std {
template<> struct hash<state_t> {
    size_t operator()(const state_t & x) const {
        return 0; // 根據具體問題實現
    }
};
}

// 方法2:函數對象,如果hash函數需要運行時數據,則用這種方法
class Hasher {
public:
    Hasher(int _m) : m(_m) {};
    size_t operator()(const state_t &s) const {
        return 0; // 根據具體問題實現
    }
private:
    int m; // 存放外面傳入的數據
};

/**
 * @brief 反向生成路徑,求一條路徑.
 * @param[in] father 樹
 * @param[in] target 目標節點
 * @return 從起點到target的路徑
 */
vector<state_t> gen_path(const unordered_map<state_t, state_t> &father,
        const state_t &target) {
    vector<state_t> path;
    path.push_back(target);

    for (state_t cur = target; father.find(cur) != father.end(); 
            cur = father.at(cur))
        path.push_back(cur);

    reverse(path.begin(), path.end());

    return path;
}

/**
 * 反向生成路徑,求所有路徑.
 * @param[in] father 存放了所有路徑的樹
 * @param[in] start 起點
 * @param[in] state 終點
 * @return 從起點到終點的所有路徑
 */
void gen_path(unordered_map<state_t, vector<state_t> > &father,
        const string &start, const state_t& state, vector<state_t> &path,
        vector<vector<state_t> > &result) {
    path.push_back(state);
    if (state == start) {
        if (!result.empty()) {
            if (path.size() < result[0].size()) {
                result.clear();
                result.push_back(path);
            } else if(path.size() == result[0].size()) {
                result.push_back(path);
            } else {
                // not possible
                throw std::logic_error("not possible to get here");
            }
        } else {
            result.push_back(path);
        }
        reverse(result.back().begin(), result.back().end());
    } else {
        for (const auto& f : father[state]) {
            gen_path(father, start, f, path, result);
        }
    }
    path.pop_back();
}
\end{Codex}


\subsubsection{求最短路徑長度或一條路徑}

\textbf{單隊列的寫法}

\begin{Codex}[label=bfs_template.cpp]
#include "bfs_common.h"

/**
 * @brief 廣搜,只用一個隊列.
 * @param[in] start 起點
 * @param[in] data 輸入數據
 * @return 從起點到目標狀態的一條最短路徑
 */
vector<state_t> bfs(state_t &start, const vector<vector<int>> &grid) {
    queue<state_t> q; // 隊列
    unordered_set<state_t> visited; // 判重
    unordered_map<state_t, state_t> father; // 樹,求路徑本身時才需要

    // 判斷狀態是否合法
    auto state_is_valid = [&](const state_t &s) { /*...*/ };

    // 判斷當前狀態是否為所求目標
    auto state_is_target = [&](const state_t &s) { /*...*/ };

    // 擴展當前狀態
    auto state_extend = [&](const state_t &s) {
        unordered_set<state_t> result;
        for (/*...*/) {
            const state_t new_state = /*...*/;
            if (state_is_valid(new_state) && 
                    visited.find(new_state) != visited.end()) {
                result.insert(new_state);
            }
        }
        return result;
    };

    assert (start.level == 0);
    q.push(start);
    while (!q.empty()) {
        // 千萬不能用 const auto&,pop() 會刪除元素,
        // 引用就變成了懸空引用
        const state_t state = q.front();
        q.pop();
        visited.insert(state);

        // 訪問節點
        if (state_is_target(state)) {
            return return gen_path(father, target); // 求一條路徑
            // return state.level + 1; // 求路徑長度
        }

        // 擴展節點
        vector<state_t> new_states = state_extend(state);
        for (const auto& new_state : new_states) {
            q.push(new_state);
            father[new_state] = state;  // 求一條路徑
            // visited.insert(state); // 優化:可以提前加入 visited 集合,
            // 從而縮小狀態擴展。這時 q 的含義略有變化,裏面存放的是處理了一半
            // 的節點:已經加入了visited,但還沒有擴展。別忘記 while循環開始
            // 前,要加一行代碼, visited.insert(start)
        }
    }

    return vector<state_t>();
    //return 0;
}
\end{Codex}


\textbf{雙隊列的寫法}
\begin{Codex}[label=bfs_template1.cpp]
#include "bfs_common.h"

/**
 * @brief 廣搜,使用兩個隊列.
 * @param[in] start 起點
 * @param[in] data 輸入數據
 * @return 從起點到目標狀態的一條最短路徑
 */
vector<state_t> bfs(const state_t &start, const type& data) {
    queue<state_t> next, current; // 當前層,下一層
    unordered_set<state_t> visited; // 判重
    unordered_map<state_t, state_t> father; // 樹,求路徑本身時才需要

    int level = -1;  // 層次

    // 判斷狀態是否合法
    auto state_is_valid = [&](const state_t &s) { /*...*/ };

    // 判斷當前狀態是否為所求目標
    auto state_is_target = [&](const state_t &s) { /*...*/ };

    // 擴展當前狀態
    auto state_extend = [&](const state_t &s) {
        unordered_set<state_t> result;
        for (/*...*/) {
            const state_t new_state = /*...*/;
            if (state_is_valid(new_state) && 
                    visited.find(new_state) != visited.end()) {
                result.insert(new_state);
            }
        }
        return result;
    };

    current.push(start);
    while (!current.empty()) {
        ++level;
        while (!current.empty()) {
            // 千萬不能用 const auto&,pop() 會刪除元素,
            // 引用就變成了懸空引用
            const auto state = current.front();
            current.pop();
            visited.insert(state);

            if (state_is_target(state)) {
                return return gen_path(father, state); // 求一條路徑
                // return state.level + 1; // 求路徑長度
            }

            const auto& new_states = state_extend(state);
            for (const auto& new_state : new_states) {
                next.push(new_state);
                father[new_state] = state;
                // visited.insert(state); // 優化:可以提前加入 visited 集合,
                // 從而縮小狀態擴展。這時 current 的含義略有變化,裏面存放的是處
                // 理了一半的節點:已經加入了visited,但還沒有擴展。別忘記 while
                // 循環開始前,要加一行代碼, visited.insert(start)
            }
        }
        swap(next, current); //!!! 交換兩個隊列
    }

    return vector<state_t>();
    // return 0;
}
\end{Codex}


\subsubsection{求所有路徑}

\textbf{單隊列}

\begin{Codex}[label=bfs_template.cpp]
/**
 * @brief 廣搜,使用一個隊列.
 * @param[in] start 起點
 * @param[in] data 輸入數據
 * @return 從起點到目標狀態的所有最短路徑
 */
vector<vector<state_t> > bfs(const state_t &start, const type& data) {
    queue<state_t> q;
    unordered_set<state_t> visited; // 判重
    unordered_map<state_t, vector<state_t> > father; // DAG

    auto state_is_valid = [&](const state_t& s) { /*...*/ };
    auto state_is_target = [&](const state_t &s) { /*...*/ };
    auto state_extend = [&](const state_t &s) {
        unordered_set<state_t> result;
        for (/*...*/) {
            const state_t new_state = /*...*/;
            if (state_is_valid(new_state)) {
                auto visited_iter = visited.find(new_state);

                if (visited_iter != visited.end()) {
                    if (visited_iter->level < new_state.level) {
                        // do nothing
                    } else if (visited_iter->level == new_state.level) {
                        result.insert(new_state);
                    } else { // not possible
                        throw std::logic_error("not possible to get here");
                    }
                } else {
                    result.insert(new_state);
                }
            }
        }

        return result;
    };

    vector<vector<string>> result;
    state_t start_state(start, 0);
    q.push(start_state);
    visited.insert(start_state);
    while (!q.empty()) {
        // 千萬不能用 const auto&,pop() 會刪除元素,
        // 引用就變成了懸空引用
        const auto state = q.front();
        q.pop();

        // 如果當前路徑長度已經超過當前最短路徑長度,
        // 可以中止對該路徑的處理,因為我們要找的是最短路徑
        if (!result.empty() && state.level + 1 > result[0].size()) break;

        if (state_is_target(state)) {
            vector<string> path;
            gen_path(father, start_state, state, path, result);
            continue;
        }
        // 必須挪到下面,比如同一層A和B兩個節點均指向了目標節點,
        // 那麼目標節點就會在q中出現兩次,輸出路徑就會翻倍
        // visited.insert(state);

        // 擴展節點
        const auto& new_states = state_extend(state);
        for (const auto& new_state : new_states) {
            if (visited.find(new_state) == visited.end()) {
                q.push(new_state);
            }
            visited.insert(new_state);
            father[new_state].push_back(state);
        }
    }

    return result;
}
\end{Codex}


\textbf{雙隊列的寫法}

\begin{Codex}[label=bfs_template.cpp]
#include "bfs_common.h"

/**
 * @brief 廣搜,使用兩個隊列.
 * @param[in] start 起點
 * @param[in] data 輸入數據
 * @return 從起點到目標狀態的所有最短路徑
 */
vector<vector<state_t> > bfs(const state_t &start, const type& data) {
    // 當前層,下一層,用unordered_set是為了去重,例如兩個父節點指向
    // 同一個子節點,如果用vector, 子節點就會在next裏出現兩次,其實此
    // 時 father 已經記錄了兩個父節點,next裏重複出現兩次是沒必要的
    unordered_set<string> current, next;
    unordered_set<state_t> visited; // 判重
    unordered_map<state_t, vector<state_t> > father; // DAG

    int level = -1;  // 層次

    // 判斷狀態是否合法
    auto state_is_valid = [&](const state_t &s) { /*...*/ };

    // 判斷當前狀態是否為所求目標
    auto state_is_target = [&](const state_t &s) { /*...*/ };

    // 擴展當前狀態
    auto state_extend = [&](const state_t &s) {
        unordered_set<state_t> result;
        for (/*...*/) {
            const state_t new_state = /*...*/;
            if (state_is_valid(new_state) && 
                    visited.find(new_state) != visited.end()) {
                result.insert(new_state);
            }
        }
        return result;
    };

    vector<vector<state_t> > result;
    current.insert(start);
    while (!current.empty()) {
        ++ level;
        // 如果當前路徑長度已經超過當前最短路徑長度,可以中止對該路徑的
        // 處理,因為我們要找的是最短路徑
        if (!result.empty() && level+1 > result[0].size()) break;

        // 1. 延遲加入visited, 這樣才能允許兩個父節點指向同一個子節點
        // 2. 一股腦current 全部加入visited, 是防止本層前一個節點擴展
        // 節點時,指向了本層後面尚未處理的節點,這條路徑必然不是最短的
        for (const auto& state : current)
            visited.insert(state);
        for (const auto& state : current) {
            if (state_is_target(state)) {
                vector<string> path;
                gen_path(father, path, start, state, result);
                continue;
            }

            const auto new_states = state_extend(state);
            for (const auto& new_state : new_states) {
                next.insert(new_state);
                father[new_state].push_back(state);
            }
        }

        current.clear();
        swap(current, next);
    }

    return result;
}
\end{Codex}

