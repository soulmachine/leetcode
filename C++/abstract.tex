\subsubsection{內容簡介}
本書的目標讀者是準備去北美找工作的碼農,也適用於在國內找工作的碼農,以及剛接觸ACM算法競賽的新手。

本書包含了 LeetCode Online Judge(\myurl{http://leetcode.com/onlinejudge})所有題目的答案,
所有代碼經過精心編寫,編碼規範良好,適合讀者反覆揣摩,模仿,甚至在紙上默寫。

全書的代碼,使用C++ 11的編寫,並在 LeetCode Online Judge 上測試通過。本書中的代碼規範,跟在公司中的工程規範略有不同,為了使代碼短(方便迅速實現):

\begindot
\item 所有代碼都是單一文件。這是因為一般OJ網站,提交代碼的時候只有一個文本框,如果還是
按照標準做法,比如分為頭文件.h和源代碼.cpp,無法在網站上提交;

\item Shorter is better。能遞歸則一定不用棧;能用STL則一定不自己實現。

\item 不提倡防禦式編程。不需要檢查malloc()/new 返回的指針是否為nullptr;不需要檢查內部函數入口參數的有效性。
\myenddot

本手冊假定讀者已經學過《數據結構》\footnote{《數據結構》,嚴蔚敏等著,清華大學出版社,
\myurl{http://book.douban.com/subject/2024655/}},
《算法》\footnote{《Algorithms》,Robert Sedgewick, Addison-Wesley Professional, \myurl{http://book.douban.com/subject/4854123/}}
這兩門課,熟練掌握C++或Java。

\subsubsection{GitHub地址}
本書是開源的,GitHub地址:\myurl{https://github.com/soulmachine/leetcode}

\subsubsection{北美求職微博羣}
我和我的小夥伴們在這裏:\myurl{http://q.weibo.com/1312378}
